%
% TCC - César Malerba - Ciência da Computação - INF/UFRGS
%
\documentclass[t]{iiufrgs}
% um tipo especfico de monografia pode ser informado como parmetro opcional:
%\documentclass[tese]{iiufrgs}
% monografias em ingls devem receber o parmetro `english':
%\documentclass[diss,english]{iiufrgs}
% a opo `openright' pode ser usada para forar incios de captulos
% em pginas mpares
% \documentclass[openright]{iiufrgs}
% para gerar uma verso somente-frente, basta utilizar a opo `oneside':
% \documentclass[oneside]{iiufrgs}
\usepackage[T1]{fontenc}        % pacote para conj. de caracteres correto
\usepackage[brazilian]{babel}
\usepackage{ae} 
\usepackage[utf8]{inputenc}   % pacote para acentuação
\usepackage{graphicx}           % pacote para importar figuras
\graphicspath{{./figs/}}
\DeclareGraphicsExtensions{.pdf,.jpg,.png}

\usepackage{times}              % pacote para usar fonte Adobe Times
\usepackage{listings}           % pacote para mostrar código fonte 
% definindo opções de listagem de código
\lstset{language=C, numbers=left, stepnumber=2, frame=single, tabsize=2}
%\usepackage{mathptmx}          % p/ usar fonte Adobe Times nas frmulas

%
% Informaes gerais
%
\title{Exploits: técnicas, detecção e prevenção}

\author{Malerba}{César}
% alguns documentos podem ter varios autores:
%\author{Flaumann}{Frida Gutenberg}
%\author{Flaumann}{Klaus Gutenberg}

% orientador e co-orientador so opcionais (no diga isso pra eles :))
\advisor[Prof.~Dr.]{Weber}{Raul Fernando}
%\coadvisor[Prof.~Dr.]{Knuth}{Donald Ervin}

% a data deve ser a da defesa; se nao especificada, so gerados
% mes e ano correntes
%\date{maio}{2001}

% o nome do curso pode ser redefinido (ex. para TCs)
\course{Ciência da Computação}

% o local de realizao do trabalho pode ser especificado (ex. para TCs)
% com o comando \location:
\location{Porto Alegre}{RS}

% itens individuais da nominata podem ser redefinidos com os comandos
% abaixo:
% \renewcommand{\nominataReit}{Prof\textsuperscript{a}.~Wrana Maria Panizzi}
% \renewcommand{\nominataReitname}{Reitora}
% \renewcommand{\nominataPRE}{Prof.~Jos{\'e} Carlos Ferraz Hennemann}
% \renewcommand{\nominataPREname}{Pr{\'o}-Reitor de Ensino}
% \renewcommand{\nominataPRAPG}{Prof\textsuperscript{a}.~Joc{\'e}lia Grazia}
% \renewcommand{\nominataPRAPGname}{Pr{\'o}-Reitora Adjunta de P{\'o}s-Gradua{\c{c}}{\~a}o}
% \renewcommand{\nominataDir}{Prof.~Philippe Olivier Alexandre Navaux}
% \renewcommand{\nominataDirname}{Diretor do Instituto de Inform{\'a}tica}
% \renewcommand{\nominataCoord}{Prof.~Carlos Alberto Heuser}
% \renewcommand{\nominataCoordname}{Coordenador do PPGC}
% \renewcommand{\nominataBibchefe}{Beatriz Regina Bastos Haro}
% \renewcommand{\nominataBibchefename}{Bibliotec{\'a}ria-chefe do Instituto de Inform{\'a}tica}
% \renewcommand{\nominataChefeINA}{Prof.~Jos{\'e} Valdeni de Lima}
% \renewcommand{\nominataChefeINAname}{Chefe do \deptINA}
% \renewcommand{\nominataChefeINT}{Prof.~Leila Ribeiro}
% \renewcommand{\nominataChefeINTname}{Chefe do \deptINT}

% A seguir so apresentados comandos especficos para alguns
% tipos de documentos.

% Relatrio de Pesquisa [rp]:
% \rp{123}             % numero do rp
% \financ{CNPq, CAPES} % orgaos financiadores

% Trabalho Individual [ti]:
% \ti{123}     % numero do TI
% \ti[II]{456} % no caso de ser o segundo TI

% Trabalho de Concluso [tc]:
% alm de definir explicitamente o nome do curso (\course) e o local
% de realizao (\location),  necessrio redefinir a nominata,
% pois as informaes necessrias dependem do curso. Ex.:
%\renewcommand{\nominata}{
%        UNIVERSIDADE FEDERAL DO RIO GRANDE DO SUL\\
%        Reitora: Prof\textsuperscript{a}.~Wrana Maria Panizzi\\
%        Pr-Reitor de Ensino: Prof.~Jos Carlos Ferraz Hennemann\\
%        Diretor do Instituto de Informtica: Prof.~Philippe Olivier Alexandre Navaux\\
%        Coordenador do curso: Prof.~Seu Creysson\\
%        Bibliotecria-chefe do Instituto de Informtica: Beatriz Regina Bastos Haro
%}

% Monografias de Especializao [espec]:
% \espec{Redes e Sistemas Distribudos}      % nome do curso
% \coord[Profa.~Dra.]{Weber}{Taisy da Silva} % coordenador do curso
% \dept{INA}                                 % departamento relacionado

%
% palavras-chave
% iniciar todas com letras minsculas, exceto no caso de abreviaturas
%
\keyword{segurança}
\keyword{teste de software}
\keyword{exploits}

%
% inicio do documento
%
\begin{document}

% folha de rosto
% s vezes  necessrio redefinir algum comando logo antes de produzir
% a folha de rosto:
% \renewcommand{\coordname}{Coordenadora do Curso}
\maketitle

% dedicatoria
\clearpage
\begin{flushright}
\mbox{}\vfill
{\sffamily\itshape
``If I have seen farther than others,\\
it is because I stood on the shoulders of giants.''\\}
--- \textsc{Sir~Isaac Newton}
\end{flushright}

% agradecimentois
\chapter*{Agradecimentos}
Estou por agradecer\ldots

% sumario
\tableofcontents

% lista de abreviaturas e siglas
% o parametro deve ser a abreviatura mais longa
\begin{listofabbrv}{SPMD}
        \item[EBP] Extended Base Pointer
		\item[ESP] Extended Stack Pointer
        \item[NUMA] Non-Uniform Memory Access
        \item[SIMD] Single Instruction Multiple Data
        \item[SPMD] Single Program Multiple Data
\end{listofabbrv}

% idem para a lista de smbolos
%\begin{listofsymbols}{$\alpha\beta\pi\omega$}
%       \item[$\sum{\frac{a}{b}}$] Somatrio do produtrio
%       \item[$\alpha\beta\pi\omega$] Fator de inconstncia do resultado
%\end{listofsymbols}

% lista de figuras
\listoffigures

% lista de tabelas
\listoftables

% resumo na lngua do documento
\begin{abstract}
a escrever\ldots
\end{abstract}

% resumo na outra língua
% como parametros devem ser passados o título e as palavras-chave
% na outra língua, separadas por vírgulas
\begin{englishabstract}{Exploits: técnicas, detecção e prevenção}{security, exploits, testing}
to be written\ldots
\end{englishabstract}

% capítulos em arquivo próprio

\chapter{Introdução}
\label{chap:introducao}

Do que se trata o trabalho?

Qual seu objetivo?

O que acrescenta?

Como é organizado?



\chapter{Conceitos iniciais}
\label{chap:conceitos_iniciais}

	\section{Exploit/Vulnerabilidade}
	O primeiro termo que devemos definir neste trabalho é exploit. Mas antes dele,
	trataremos de vulnerabilidade - pois eles têm uma ligação estreita.
	Podemos definir vulnerabilidade como uma falha em um sistema que permite
	a um atacante usá-lo de uma forma não prevista pelo projetista \cite{Anley2007}.
	Ou seja, uma vulnerabilidade implica a possibilidade de uso indevido de um sistema.
	Os passos necessários para explorar essa fraqueza, ou mesmo o código (programa) que pode tirar
	proveito da vulnerabilidade é descrito como exploit.
	Um exploit surge apenas quando há uma vulnerabilidade - mas podem existir
	vulnerabilidades para as quais não exista exploit.


	\section{Conceitos básicos}
	Neste trabalho iremos tratar de exploits na arquitetura x86 de 32 bits. Trata-se da arquitetura de computadores
	pessoais mais difundida nos dias de hoje. Mas boa parte do estudo realizado pode ser aplicada
	a praticamente qualquer outra arquitetura.

	\section{Gerência de memória}
	O controle da memória é um ponto crítico. Falhas nele acabam resultando em vulnerabilidades 
	gravíssimas. Faremos uma breve abordagem sobre o gerenciamento de memória sobre
	o ponto de vista dos exploits.

	Um primeiro ponto a destacar sobre a memória é um princípio básico que norteia
	quase todas as arquiteturas modernas. Dados e instruções não são diferenciados na memória.
	Ou seja, não há uma separação rígida entre instruções que compõem um programa e os dados
	sobre os quais opera. Essa característica foi herdada da arquitetura básica de von Neumann.
	Como veremos a seguir, essa decisão de design, com origem nos anos 1940, embora tenha
	facilitado a evolução dos computadores, abriu caminhos para os exploits que conhecemos hoje. 

	Abaixo descrevemos o layout básico da memória de um processo em um sistema UNIX.
	Ele pode ser separado em 6 partes fundamentais:
	\begin{description}
		\item[text]
			A parte que contém as instruções do programa - seu código propriamente dito.
			Seu tamanho é fixo durante a execução e ela não deve possibilitar escrita.
		\item[data]
			Contém variáveis globais já inicializadas. Seu tamanho é fixo durante a execução.
		\item[bss]
			Nome de origem história significando Block Started by Symbol. Área da memória responsável
			por armazenar variáveis globais	não inicializadas. Como text e data, bss também tem tamanho 
			fixo conhecido desde o início do processo. 
		\item[Heap]
			Espaço para variáveis alocadas dinamicamente. A chamada de sistema sbrk é responsável
			pelo controle do crescimento/encolhimento desse espaço. Bibliotecas geralmente facilitam a vida
			do programador disponibilizando interfaces mais amigáveis como malloc() e free(). Assim a biblioteca
			se encarrega de chamar sbrk() para diminuir/aumentar o Heap. Ela cresce do endereço mais baixo para o
			mais alto.
		\item[Stack]
			Mantém controle das chamadas de funções. Possibilita a recursividade. Logo, possui
			tamanho variável - crescendo do endereço mais alto para o mais baixo (sendo antagonista do Heap - ver
			figura \ref{fig:regioes_memoria}). 
			Esse crescimento é que torna possível que uma chamada de função que tenha seus dados
			sobrescritos influencie numa chamada de função anterior. Esse é o princípio do buffer overflow - tratado
			posteriormente.
		\item[Enviroment]
			A última porção de memória do processo guarda uma cópia das variáveis de ambiente do sistema.
			Essa seção possui permissão de escrita, mas como bss, data e text, possui tamanho fixo.
	\end{description}

	\begin{figure}
		\begin{center}
		\includegraphics[width=0.45\textwidth]{regioes_memoria.pdf}
		\caption{Regiões de memória em um processo.}
		\label{fig:regioes_memoria}
		\end{center}
	\end{figure}

	\section{Funcionamento mais detalhado do Stack}
	A pilha é uma região contínua com base fixa e tamanho variável.
	Na arquitetura abordada por esse trabalho, x86 (bem como em muitas outras), a pilha cresce
	em direção ao endereço mais baixo. É organizada em \textsl{frames} que são os blocos
	alocados quando ocorrem chamadas a funções. Cada \textsl{frame} contém(ver figura \ref{fig:stack_frame}):
	\begin{itemize}
		\item parâmetros
		\item variáveis locais
		\item endereço de retorno da função anterior
		\item endereço do \textsl{frame} da função que a chamou
	\end{itemize}

	\begin{figure}
		\begin{center}
		\includegraphics[width=0.5\textwidth]{stack_frame_furlan.jpg}
		\caption{Organização do \textsl{frame} na pilha. Retirado de \cite{Furlan2005} pg. 17.}
		\label{fig:stack_frame}
		\end{center}
	\end{figure}

	\subsection{Chamada de funções}
	Quando uma função é chamada, seus parâmetros são empilhadas e posteriormente o endereço
	do retorno. Isso fica a encargo da função que faz a chamada.
	Para completar o \textsl{frame}, aquela que é chamada, empilha o endereço do frame da função chamadora
	(EBP) e posteriormente aloca na pilha o espaço correspondente a suas variáveis locais.
	É importante ressaltar que, caso o endereço de retorno, empilhado por quem chama, seja alterado,
	o fluxo de execução é mudado. Pois é justamente este o princípio do \textsl{buffer overflow}.

	\section{Funcionamento mais detalhado do Heap}
	A porção de memória correspondente ao heap possibilita ao programador alocar dinamicamente memória
	que fica disponível durante toda a execução para qualquer chamada de função. Diferentemente
	da memória alocada no stack - que é perdida quando a função retorna.

	\section{Registradores de controle}
	Uma parte fundamental da arquitetura que deve ser mencionada são os registradores que possuem
	relação direta com o gerenciamento da memória.
	Talvez o mais importante (na arquitetura base do estudo IA32) seja o EIP(Extended Instruction Pointer).
	Ele indica o endereço da próxima instrução. Sobrescrevê-lo equivale obter o controle
	do fluxo de um processo.
	Além dele, destacamos EBP(Extended Base Pointer) e ESP(Extended Stack Pointer).
	ESP indica o endereço do último valor inserido na pilha.
	O EBP indica o início da pilha para aquela chamada de função. É usado para referenciar variáveis
	locais da função.




\chapter{Classificação de vulnerabilidades}
\label{chap:classificacao}

	A classificação de vulnerabilidades representa enorme desafio.
	Nos dias de hoje, não existe nenhum padrão aceito globalmente para essa tarefa.
	Ainda assim, já houve vários avanços na área. 
	Existem padrões para enumerar e catalogar vulnerabilidades, bem como propostas
	que podem criar bases para uma classificação que venha a ser aceita pela comunidade.
	Métricas, relativas	à gravidade e ao impacto, também estão disponíveis
	e são empregadas no auxílio às instituições nas tomadas	de decisões.

	
	No trabalho de Seacord e Householder, \cite{Seacord2005}, temos os fatores que motivam a
	busca pela organização das vulnerabilidades em classes:
	\begin{itemize}
		\item{O entendimento das ameaças que elas representam.}
		\item{Correlacionamento de incidentes, de \textsl{exploits} e de artefatos.}
		\item{Avaliação da efetividade das ações de defesa.}
		\item{Descoberta de tendências de vulnerabilidades.}
	\end{itemize}

	
	Vemos, portanto, que a taxonomia\footnote{Ciência da classificação.} das vulnerabilidades
	pode trazer uma série de benefícios para seu entendimento, tratamento e prevenção.
	Nesse capítulo, nosso intuito é abordar a dificuldade nesse processo e apresentar
	os avanços já obtidos nesse sentido.  


	\section{A dificuldade em classificar; estágio já alcançado: enumeração}
		Antes de entrarmos no mérito das vulnerabilidades, é preciso definir
		com precisão dois termos que utilizaremos por todo o capítulo: classificar e enumerar.
		Como veremos, a taxonomia é mais custosa que a enumeração.

		\subsection{Classificar}
			\label{subsec:classificar}
			Como podemos encontrar em \cite{Holanda1975}, classificar implica "distribuir em classes e/ou grupos
			segundo um sistema". Logo, para a classificação, é preciso haver uma metodologia que possa
			separar os itens em estudo em diferentes grupos. A ciência que estuda esse processo
			é chamada taxonomia. Ela é guiada, conforme \cite{Gregio2005_1}, pelos princípios taxonômicos.
			São eles:
			\begin{description}
				\item[Exclusão mútua]
					Um item não podem ser categorizado simultaneamente em dois grupos.
				\item[Exaustividade]
					Os grupos, unidos, incluem todas as possibilidades.
				\item[Repetibilidade]
					Diferentes pessoas extraindo a mesma característica do objeto devem concordar com
					o valor observado.
				\item[Aceitabilidade]
					Os critérios devem ser lógicos e intuitivos para serem aceitos pela comunidade.
				\item[Utilidade]
					A classificação pode ser utilizada na obtenção de conhecimento na área de pesquisa.
			\end{description}

			
			Vemos que os critérios para a taxonomia são exigentes e pressupõem uma metodologia
			cuidadosamente gerada para atendê-los. 

		\subsection{Enumerar}
			A enumeração é um processo semelhante a 
			"indicar por números; relacionar metodicamente"; como encontramos
			em \cite{Holanda1975}.
			Trata-se, portanto, de algo muito mais simples que a classificação.
			Mesmo sendo mais simples, é extremamente importante pois permite
			que os itens enumerados sejam facilmente apontados e diferenciados entre si.
			
			
			Sem um procedimento de enumeração dos objetos de estudo, adotado de comum acordo,
			não é possível que duas partes se comuniquem sem risco de cometerem enganos. 
			Quem garante que estão tratando exatamente da mesma coisa naquele momento?
			Logo a enumeração é essencial para o devido entendimento sobre os objetos
			de estudo.

		\subsection{Da enumeração à classificação}
			No trabalho de Mann, \cite{Mann1999}, há um excelente paralelo entre a
			questão abordada nesse capítulo e o advento da tabela 
			periódica\footnote{Dispõe sistematicamente os elementos de acordo com suas propriedades permitindo
			uma análise multidimensional.} na Química. 
			A organização dos elementos da forma como conhecemos hoje na tabela periódica
			foi um processo longo que culminou com as ideias de Dimitri Mendeleev.
			Outros químicos que o precederam foram responsáveis pela identificação e
			listagem dos elementos. Isso possibilitou um melhor estudo e uma maior
			troca de informação precisa entre os pesquisadores.


			Segundo Mann, a tabela periódica só pode ser efetivamente criada graças
			aos esforços daqueles que enumeraram os elementos de forma mais simples
			antes de Mendeleev. O trabalho deles permitiu a interoperabilidade
			necessária para o surgimento da tabela periódica.
			Da mesma forma, nos anos antecedentes a 2000, a comunidade que estudava
			e acompanhava as vulnerabilidades estava num patamar semelhante àqueles
			que precederam Mendeleev. Ou seja, sequer havia uma enumeração mais
			amplamente aceita e reconhecida das vulnerabilidades que permitisse
			avanços suficientes para uma taxonomia.

			
			Citamos o ano de 2000 como parâmetro, pois nessa época, 1999, surgiria um projeto
			que se tornaria referência para a criação de uma padronização da enumeração
			de vulnerabilidades. Não seria ainda um evento comparável à criação da
			tabela periódica para Química (pois não trouxe a taxonomia) 
			mas certamente lançaria as bases para
			a interoperabilidade exigida para estudos mais aprofundados na área.
			Estamos falando da criação do 
			CVE(Common Vulnerabilities and Exposures)\footnote{http://cve.mitre.org}\footnote{Na época
			de sua criação era originalmente conhecido por Common Vulnerabilities Enumeration - vide
			\cite{Meunier2006} pg. 9.}
			pelo MITRE. A seção \ref{sec:cve} traz mais detalhes.


			Podemos dizer, portanto, que atualmente, embora não tenhamos uma taxonomia
			amplamente aceita pela comunidade, já foi atingido o estágio de enumeração.
			Projetos como o CVE podem ser considerados como marcos dessa etapa.
			A seguir, iremos abordar em mais detalhes o surgimento e o funcionamento dele.
			Isso nos possibilitará compreender melhor a complexidade da classificação
			das vulnerabilidades bem como irá facilitar o entendimento dos capítulos
			seguintes que abordam \textsl{exploits}.
			

	\section{CVE}
	\label{sec:cve}

		\subsection{Surgimento e objetivos}
			Para deixar mais nítida a dificuldade de interoperabilidade
			das organizações no que se refere a ameaças de segurança na época
			que antecede o CVE, segue abaixo uma tabela, extraída de \cite{Martin2001}.
			A tabela \ref{tab:torre_babel}, mostra
			como diferentes organizações se referiam à mesma vulnerabilidade
			em 1998. Trata-se de um verdadeira Torre de Babel.

			\begin{table}
				\begin{tabular}{|l|c|c|c|c|}
					\hline
						\textbf{Organização} & \textbf{Como se referia à vulnerabilidade}\\
					\hline
						CERT\footnotemark[1] & CA-96.06.cgi\_example\_code\\
					\hline
						Cisco Systems\footnotemark[2] & http - cgi-phf\\
					\hline
						DARPA & 0x00000025 = http PHF attack\\	
					\hline
						IBM ERS & ERS-SVA-E01-1996:002.1\\	
					\hline
						Security Focus\footnotemark[3] & \#629 - phf Remote Command Execution Vulnerability\\	
					\hline
				\end{tabular}
				\footnotetext[1]{site}
				\footnotetext[2]{site}
				\footnotetext[3]{site}
				\caption{Uma vulnerabilidade: diversos nomes e nenhum entendimento}\label{tab:torre_babel}
			\end{table}
			

			O CVE, como dito anteriormente, surge em 1999 e seu maior objetivo, como podemos
			ler em sua FAQ, \cite{CVE2010}, é tornar mais fácil
			o compartilhamento de informações sobre vulnerabilidades utilizando uma enumeração comum.
			Essa enumeração é realizada através da manutenção de uma lista na qual, conforme
			encontramos em \cite{Santos2004}, valem os seguintes princípios:
			\begin{itemize}
				\item{Atribuição de um nome padronizado e único a cada vulnerabilidade.}
				\item{Independência das diferentes perspectivas em que a vulnerabilidade ocorre.}
				\item{Abertura total voltada ao compartilhamento pleno das informações.}
			\end{itemize}


			Segundo a própria organização, vide \cite{CVE2010}, o CVE não possui um objetivo 
			inicial de conter alguma espécie de taxonomia. Essa é considerada uma área de pesquisa
			ainda em desenvolvimento. É esperado que, com o auxílio prestado pela
			catalogação das vulnerabilidades já constitua um importante passo para que
			isso ocorra.

			\begin{figure}
				\begin{center}
					\includegraphics[width=0.65\textwidth]{vulnerabilidades_CVE.jpg}
					\caption{Vulnerabilidades registradas no CVE ao longo do tempo - retirada 
							de \cite{Florian2009}}
					\label{fig:vulnerabilidades_CVE}
				\end{center}
			\end{figure}

			
			Na figura \ref{fig:vulnerabilidades_CVE}, podemos visualizar um histórico
			da quantidade de vulnerabilidades adicionadas. Nos últimos anos podemos
			perceber que os incidentes registrados ficam na média de 7000.
			Isso mostra relevância que o projeto do CVE alcançou.
			
			 
		
		\subsection{Funcionamento}
			O CVE é formado por uma junta de especialistas em segurança dos meios acadêmico, comercial e
			governamental. Eles são responsáveis por analisar e definir o que será feito dos reports passados
			pela comunidade - se eles devem ou não se integrar àqueles já pertencentes à lista.
			Cabe a eles definir nome, descrição e referências para cada nova ameaça.


			Esse processo inicia quando uma vulnerabilidade é reportada.
			Ela assume um CAN(Candidate Number), número de candidata.
			Até que ela seja adicionada à lista, ele permanece com um CAN que
			a identificará. Apenas após o devido estudo e aprovação do caso pela junta
			responsável, é que ela assume um identificador CVE.

			
			Os identificadores CVE são definidos conforme o padrão: CVE-2010-0021.
			Onde, separados por '-', há 3 partes. A primeira é fixa: CVE.
			A segunda refere-se ao ano de surgimento; enquanto a terceira
			indica o número sequencial daquela vulnerabilidade entre todas
			aquelas que foram adicionadas naquele ano. Logo, no exemplo fornecido,
			essa seria a vigésima primeira de 2010.
		
			
			Uma vez integrada, a vulnerabilidade passa a estar publicamente disponível.
			Essa abertura pode servir de auxílio aos atacantes - pois informações
			sobre possíveis furos de segurança são sempre bem vindas a eles.
			Porém, conforme podemos verificar na FAQ do CVE, \cite{CVE2010}, há uma série
			de motivos pelos quais a disponibilidade desses dados supera o risco
			oferecido pela exposição. São eles:
			\begin{itemize}
				\item{O CVE está restrito a publicar vulnerabilidades já conhecidas.}
				\item{Por diversas razões, a comunidade de segurança de informação
					sofre mais para compartilhar dados sobre as ameças
					que os atacantes.}
				\item{É muito mais custo a uma organização proteger toda sua rede
					contra as ameças que a um atacante descobrir e explorar uma delas para
					comprometer alguma das redes.}
			\end{itemize}
			
			

		
		
		
	\section{Propostas taxonômicas}
		Como foi referenciado anteriormente, não há padrão amplamente aceito
		para a taxonomia de vulnerabilidades.
		Mesmo assim, há diversas propostas que buscam resolver essa questão em aberto.
		
		
		O trabalho de \cite{Gregio2005} faz um levantamento amplo de algumas
		dessas alternativas. Mas, como veremos, nenhuma delas atinge os objetivos
		de uma taxonomia plena - apresentados na seção \ref{subsec:classificar}.
		Assim, buscaremos discutir as ideias para as metodologias
		de classificação com o intuito de apontar suas vantagens e fraquezas.

	\section{Métricas para vulnerabilidades}



\chapter{NULL pointer Exploit}
\label{chap:null_pointer_exploit}

	Dentre as várias técnicas de exploits existentes, uma que certamente merece
	destaque, é o NULL pointer exploit.
	Surgiu recentemente e é fruto da crescente dificuldade em aplicar técnicas
	que exploram vulnerabilidades de corrupção de memória.
	Em princípio, deferenciar um ponteiro nulo não é considerada
	uma vulnerabilidade explorável, mas, como veremos, isso nem sempre é verdade.

	
	O ano de 2009 chegou a ser considerado o ano do "kernel NULL pointer deference"
	em virtude da grande quantidade de falhas desse gênero encontradas no kernel do Linux.
	(citar http://cwe.mitre.org/top25/ para o ano de 2009)
	Vemos então que essa falha chegou a atingir notoriedade pelo grande impacto que causou.
	Nossa intenção é apresentar esse tipo de exploit, mostrar como pode ocorrer, ilustrando
	com exemplos e apontando formas de prevenção.
	
	
	
	\section{O que é um NULL pointer}
		O primeiro ponto a ser abordado é o NULL pointer.
		Na linguagem de programação C, podemos considerar um ponteiro como um valor inteiro
		que referencia uma posição de memória. Ou seja, trata-se de um valor que aponta
		para um determinado ponto no espaço de endereçamento. Quando um ponteiro é deferenciado,
		passamos a acessar o valor presente na posição de memória para o qual ele aponta.
		Ilustrando, segue pequeno trecho de código C.
		\begin{lstlisting}[label=pointer_example,caption=Ponteiro em C]
int val = 10;
int *pointer = &value;
/* pointer has the address of val */
/* *pointer returns the value stored in val */
		\end{lstlisting}
		

		No Linux, o arquivo stddef.h contém a definição de NULL, que por
		convenção, denomina um ponteiro com valor zero.
		Um ponteiro nulo, então, aponta para a posição zero de memória. 
		Como, em regra geral, os sistemas utilizam o esquema de memóra virtual,
		na prática, esse endereço zero deve ser considerado tão somente no espaço
		de endereçamento do processo em questão.
		Como normalmente ele não constitui um mapeamento válido, pois os processos não
		iniciam com aquela porção mapeada, os acessos a essa região implicam violação
		às regras do esquema de memória virtual. Erros desse gênero resultam em
		segmentation fault.   

		Logo, quando deferenciamos um ponteiro nulo, temos um acesso não previsto
		que configura uma violação 
		Por essa razão, que o acesso a um ponteiro com valor
		zero acaba sendo um erro na aplicação que resulta em sua finalização.
		
		Embora o foco do trabalho recaia sobre a arquitetura x86, é válido identificar a repercussão
		de uma acesso a posição zero de memória em outros casos.
		Existem arquiteturas nas quais o endereço zero já é mapeado inicialmente. 
		Podemos apontar o caso da ARM e da XScale; ambas para sistemas embarcados. 
		Nelas, o vetor de exceções se encontra nessa posição. Ele contém
		endereços para o tratamento de, por exemplo, interrupções de hardware e software.
		Naturalmente, um acesso a um ponteiro nulo nesses casos constitui enorme
		problema em termos de segurança.

		

	\section{Como funciona a técnica}

	\section{Exemplos}

	\subsection{Flash Player exploit}
	
	\subsection{Falhas no kernel do Linux}
		Existem diversas falhas no kernel do Linux relacionadas a NULL pointer.
		Desde problemas na inicialização de estruturas de dados até a erros na compilação.		
	
	\subsection{Pidgin}

	\section{Proteções}


\chapter{Heap overflow}
\label{chap:heap_overflow}

\chapter{Programação segura: evitando vulnerabilidades no desenvolvimento}
\label{chap:heap_overflow}


\chapter{Fuzzing: detecção de vulnerabilidades}
\label{chap:fuzzing}

	\section{O que é fuzzing?}

	\section{Fuzzing aplicado ao teste de software}

	\section{Detectando vulnerabilidades}

	\section{Exemplos}



\chapter{Conclusão}
\label{chap:conclusao}
	Nesse trabalho foram abordados aspectos essenciais relacionados à segurança do software.
	Ao tratar de vulnerabilidades e técnicas de \textsl{exploits}, ele objetivou
	trazer ao leitor um contexto fundamental para um entendimento da área. Dada a relevância
	que o software atingiu nos dias de hoje, não é mais admissível que qualquer desenvolvimento
	sério desconsidere princípios de segurança. Sendo eles: as possíveis vulnerabilidades,
	as formas de ataque e, naturalmente, as formas de prevenção.


	No que se refere às vulnerabilidades, esse trabalho, ao tratar de sua classificação, pode identificar
	que esse tópico ainda não é pacífico no meio acadêmico ou industrial. Mesmo que tenham
	sidos feitos avanços, a comunidade carece de um padrão aceito uniformemente. Ficou nítido que a complexidade
	dessa tarefa é enorme. O próprio caráter multifacetado das vulnerabilidades explica um pouco essa barreira;
	elas podem ser analisados por diversos ângulos e estamos longe de encontrar uma visão unificadora
	que traga sentido a todas suas faces. Apenas assim seria alcançada uma taxonomia em sentido estrito.


	No campo dos \textsl{exploits}, pode ser visto que, com a evolução natural
	das técnicas de ataque e de defesa, a vida dos especialistas na área torna-se cada vez mais árdua.
	As formas mais simples de explorar vulnerabilidades já não são mais efetivas; seja porque as falhas
	que as tornam possíveis ficaram menos frequentes no desenvolvimento, seja porque proteções mais
	bem concebidas foram sendo habilitadas por padrão nos sistemas. Isso vai obrigando os atacantes
	a encontrarem novos métodos cada vez mais sofisticados que, naturalmente, vão exigindo conhecimento
	ainda mais específico. Em alguns momentos, porém, ainda será possível surgir alguma espécie
	de reviravolta - como na descoberta dos diversos erros de \textsl{NULL pointer} no kernel do Linux.
	Episódio que demonstrou a existência de uma série de falhas por vários anos em um dos sistemas
	mais utilizados - surpreendentemente, algumas delas de fácil exploração.


	Para a prevenção de problemas de segurança no software, uma das principais propostas apresentadas foi
	o uso do testes fuzzing. Sendo uma técnica extremamente eficiente e que já vem sendo usada
	por atacantes para a descoberta de problemas nos sistemas, esse trabalho buscou demonstrar seu valor
	e indicá-la como arma a ser utilizada pelos próprios desenvolvedores. Por que já não conceber, desde
	o princípio, um projeto considerando essa alternativa de teste se os atacantes certamente irão utilizá-la?
	Conforme foi visto, gigantes da área, como a Microsoft, já perceberam seu enorme valor e investem
	fortemente nela. É necessário, portanto, que, ao menos, consideremos essa possibilidade. Isso
	porque não é aceitável correr o risco de deixar apenas para os atacantes utilizarem e aperfeiçoarem
	uma técnica que possa desequilibrar em favor deles.



%
% referencias
%
\bibliographystyle{abnt}
\bibliography{bibliografia}

\end{document}
