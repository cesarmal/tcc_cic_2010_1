%
% TCC - César Malerba - Ciência da Computação - INF/UFRGS
%
\documentclass[t]{iiufrgs}
% um tipo especfico de monografia pode ser informado como parmetro opcional:
%\documentclass[tese]{iiufrgs}
% monografias em ingls devem receber o parmetro `english':
%\documentclass[diss,english]{iiufrgs}
% a opo `openright' pode ser usada para forar incios de captulos
% em pginas mpares
% \documentclass[openright]{iiufrgs}
% para gerar uma verso somente-frente, basta utilizar a opo `oneside':
% \documentclass[oneside]{iiufrgs}
\usepackage[T1]{fontenc}        % pacote para conj. de caracteres correto
\usepackage[brazilian]{babel}
\usepackage{ae} 
\usepackage[utf8]{inputenc}   % pacote para acentuação
\usepackage{graphicx}           % pacote para importar figuras
\graphicspath{{./figs/}}
\DeclareGraphicsExtensions{.pdf,.jpg,.png}
\usepackage{multirow} % multiplas linhas nas tabelas

\usepackage{times}              % pacote para usar fonte Adobe Times
\usepackage{listings}           % pacote para mostrar código fonte 
% definindo opções de listagem de código
\lstset{language=C, numbers=left, stepnumber=1, frame=single, tabsize=2, breaklines=true}
%\usepackage{mathptmx}          % p/ usar fonte Adobe Times nas fórmulas

%
% Informaes gerais
%
\title{Exploits: técnicas, detecção e prevenção}

\author{Malerba}{César}
% alguns documentos podem ter varios autores:
%\author{Flaumann}{Frida Gutenberg}
%\author{Flaumann}{Klaus Gutenberg}

% orientador e co-orientador so opcionais (no diga isso pra eles :))
\advisor[Prof.~Dr.]{Weber}{Raul Fernando}
%\coadvisor[Prof.~Dr.]{Knuth}{Donald Ervin}

% a data deve ser a da defesa; se nao especificada, so gerados
% mes e ano correntes
%\date{maio}{2001}

% o nome do curso pode ser redefinido (ex. para TCs)
\course{Ciência da Computação}

% o local de realizao do trabalho pode ser especificado (ex. para TCs)
% com o comando \location:
\location{Porto Alegre}{RS}

% itens individuais da nominata podem ser redefinidos com os comandos
% abaixo:
% \renewcommand{\nominataReit}{Prof\textsuperscript{a}.~Wrana Maria Panizzi}
% \renewcommand{\nominataReitname}{Reitora}
% \renewcommand{\nominataPRE}{Prof.~Jos{\'e} Carlos Ferraz Hennemann}
% \renewcommand{\nominataPREname}{Pr{\'o}-Reitor de Ensino}
% \renewcommand{\nominataPRAPG}{Prof\textsuperscript{a}.~Joc{\'e}lia Grazia}
% \renewcommand{\nominataPRAPGname}{Pr{\'o}-Reitora Adjunta de P{\'o}s-Gradua{\c{c}}{\~a}o}
% \renewcommand{\nominataDir}{Prof.~Philippe Olivier Alexandre Navaux}
% \renewcommand{\nominataDirname}{Diretor do Instituto de Inform{\'a}tica}
% \renewcommand{\nominataCoord}{Prof.~Carlos Alberto Heuser}
% \renewcommand{\nominataCoordname}{Coordenador do PPGC}
% \renewcommand{\nominataBibchefe}{Beatriz Regina Bastos Haro}
% \renewcommand{\nominataBibchefename}{Bibliotec{\'a}ria-chefe do Instituto de Inform{\'a}tica}
% \renewcommand{\nominataChefeINA}{Prof.~Jos{\'e} Valdeni de Lima}
% \renewcommand{\nominataChefeINAname}{Chefe do \deptINA}
% \renewcommand{\nominataChefeINT}{Prof.~Leila Ribeiro}
% \renewcommand{\nominataChefeINTname}{Chefe do \deptINT}

% A seguir so apresentados comandos especficos para alguns
% tipos de documentos.

% Relatrio de Pesquisa [rp]:
% \rp{123}             % numero do rp
% \financ{CNPq, CAPES} % orgaos financiadores

% Trabalho Individual [ti]:
% \ti{123}     % numero do TI
% \ti[II]{456} % no caso de ser o segundo TI

% Trabalho de Concluso [tc]:
% alm de definir explicitamente o nome do curso (\course) e o local
% de realizao (\location),  necessrio redefinir a nominata,
% pois as informaes necessrias dependem do curso. Ex.:
%\renewcommand{\nominata}{
%        UNIVERSIDADE FEDERAL DO RIO GRANDE DO SUL\\
%        Reitora: Prof\textsuperscript{a}.~Wrana Maria Panizzi\\
%        Pr-Reitor de Ensino: Prof.~Jos Carlos Ferraz Hennemann\\
%        Diretor do Instituto de Informtica: Prof.~Philippe Olivier Alexandre Navaux\\
%        Coordenador do curso: Prof.~Seu Creysson\\
%        Bibliotecria-chefe do Instituto de Informtica: Beatriz Regina Bastos Haro
%}

% Monografias de Especializao [espec]:
% \espec{Redes e Sistemas Distribudos}      % nome do curso
% \coord[Profa.~Dra.]{Weber}{Taisy da Silva} % coordenador do curso
% \dept{INA}                                 % departamento relacionado

%
% palavras-chave
% iniciar todas com letras minsculas, exceto no caso de abreviaturas
%
\keyword{segurança}
\keyword{teste de software}
\keyword{exploits}

%
% inicio do documento
%
\begin{document}

% folha de rosto
% \renewcommand{\coordname}{Coordenadora do Curso}
\maketitle

% dedicatoria
\clearpage
\begin{flushright}
\mbox{}\vfill
{\sffamily\itshape
``If I have seen farther than others,\\
it is because I stood on the shoulders of giants.''\\}
--- \textsc{Sir~Isaac Newton}
\end{flushright}

% agradecimentois
\chapter*{Agradecimentos}
Estou por agradecer\ldots

% sumario
\tableofcontents

% lista de abreviaturas e siglas
% o parametro deve ser a abreviatura mais longa
\begin{listofabbrv}{SPMD}
        \item[EBP] Extended Base Pointer
		\item[ESP] Extended Stack Pointer
		\item[CVE] Common Vulnerabilities and Exposures
		\item[CVSS] Common Vulnerability Scoring System
		\item[CPU] Central Processing Unit
		\item[CWE] Common Weakness Enumeration
		\item[DoS] Denial of Service
		\item[DPL] Descriptor Privelege Level
        \item[NUMA] Non-Uniform Memory Access
        \item[RAM] Random Access Memory
        \item[RPC] Remote Procedure Call
        \item[SWF] Shockwave Flash
\end{listofabbrv}

% idem para a lista de smbolos
%\begin{listofsymbols}{$\alpha\beta\pi\omega$}
%       \item[$\sum{\frac{a}{b}}$] Somatrio do produtrio
%       \item[$\alpha\beta\pi\omega$] Fator de inconstncia do resultado
%\end{listofsymbols}

% lista de figuras
\listoffigures

% lista de tabelas
\listoftables

% resumo na lngua do documento
\begin{abstract}
a escrever\ldots
\end{abstract}

% resumo na outra língua
% como parametros devem ser passados o título e as palavras-chave
% na outra língua, separadas por vírgulas
\begin{englishabstract}{Exploits: técnicas, detecção e prevenção}{security, exploits, testing}
to be written\ldots
\end{englishabstract}

% capítulos em arquivo próprio

\chapter{Introdução}
\label{chap:introducao}

Do que se trata o trabalho?

Qual seu objetivo?

O que acrescenta?

Como é organizado?



\chapter{Conceitos iniciais}
\label{chap:conceitos_iniciais}

	\section{Exploit/Vulnerabilidade}
	O primeiro termo que devemos definir neste trabalho é exploit. Mas antes dele,
	trataremos de vulnerabilidade - pois eles têm uma ligação estreita.
	Podemos definir vulnerabilidade como uma falha em um sistema que permite
	a um atacante usá-lo de uma forma não prevista pelo projetista \cite{Anley2007}.
	Ou seja, uma vulnerabilidade implica a possibilidade de uso indevido de um sistema.
	Os passos necessários para explorar essa fraqueza, ou mesmo o código (programa) que pode tirar
	proveito da vulnerabilidade é descrito como exploit.
	Um exploit surge apenas quando há uma vulnerabilidade - mas podem existir
	vulnerabilidades para as quais não exista exploit.


	\section{Conceitos básicos}
	Neste trabalho iremos tratar de exploits na arquitetura x86 de 32 bits. Trata-se da arquitetura de computadores
	pessoais mais difundida nos dias de hoje. Mas boa parte do estudo realizado pode ser aplicada
	a praticamente qualquer outra arquitetura.

	\section{Gerência de memória}
	O controle da memória é um ponto crítico. Falhas nele acabam resultando em vulnerabilidades 
	gravíssimas. Faremos uma breve abordagem sobre o gerenciamento de memória sobre
	o ponto de vista dos exploits.

	Um primeiro ponto a destacar sobre a memória é um princípio básico que norteia
	quase todas as arquiteturas modernas. Dados e instruções não são diferenciados na memória.
	Ou seja, não há uma separação rígida entre instruções que compõem um programa e os dados
	sobre os quais opera. Essa característica foi herdada da arquitetura básica de von Neumann.
	Como veremos a seguir, essa decisão de design, com origem nos anos 1940, embora tenha
	facilitado a evolução dos computadores, abriu caminhos para os exploits que conhecemos hoje. 

	Abaixo descrevemos o layout básico da memória de um processo em um sistema UNIX.
	Ele pode ser separado em 6 partes fundamentais:
	\begin{description}
		\item[text]
			A parte que contém as instruções do programa - seu código propriamente dito.
			Seu tamanho é fixo durante a execução e ela não deve possibilitar escrita.
		\item[data]
			Contém variáveis globais já inicializadas. Seu tamanho é fixo durante a execução.
		\item[bss]
			Nome de origem história significando Block Started by Symbol. Área da memória responsável
			por armazenar variáveis globais	não inicializadas. Como text e data, bss também tem tamanho 
			fixo conhecido desde o início do processo. 
		\item[Heap]
			Espaço para variáveis alocadas dinamicamente. A chamada de sistema sbrk é responsável
			pelo controle do crescimento/encolhimento desse espaço. Bibliotecas geralmente facilitam a vida
			do programador disponibilizando interfaces mais amigáveis como malloc() e free(). Assim a biblioteca
			se encarrega de chamar sbrk() para diminuir/aumentar o Heap. Ela cresce do endereço mais baixo para o
			mais alto.
		\item[Stack]
			Mantém controle das chamadas de funções. Possibilita a recursividade. Logo, possui
			tamanho variável - crescendo do endereço mais alto para o mais baixo (sendo antagonista do Heap - ver
			figura \ref{fig:regioes_memoria}). 
			Esse crescimento é que torna possível que uma chamada de função que tenha seus dados
			sobrescritos influencie numa chamada de função anterior. Esse é o princípio do buffer overflow - tratado
			posteriormente.
		\item[Enviroment]
			A última porção de memória do processo guarda uma cópia das variáveis de ambiente do sistema.
			Essa seção possui permissão de escrita, mas como bss, data e text, possui tamanho fixo.
	\end{description}

	\begin{figure}
		\begin{center}
		\includegraphics[width=0.45\textwidth]{regioes_memoria.pdf}
		\caption{Regiões de memória em um processo.}
		\label{fig:regioes_memoria}
		\end{center}
	\end{figure}

	\section{Funcionamento mais detalhado do Stack}
	A pilha é uma região contínua com base fixa e tamanho variável.
	Na arquitetura abordada por esse trabalho, x86 (bem como em muitas outras), a pilha cresce
	em direção ao endereço mais baixo. É organizada em \textsl{frames} que são os blocos
	alocados quando ocorrem chamadas a funções. Cada \textsl{frame} contém(ver figura \ref{fig:stack_frame}):
	\begin{itemize}
		\item parâmetros
		\item variáveis locais
		\item endereço de retorno da função anterior
		\item endereço do \textsl{frame} da função que a chamou
	\end{itemize}

	\begin{figure}
		\begin{center}
		\includegraphics[width=0.5\textwidth]{stack_frame_furlan.jpg}
		\caption{Organização do \textsl{frame} na pilha. Retirado de \cite{Furlan2005} pg. 17.}
		\label{fig:stack_frame}
		\end{center}
	\end{figure}

	\subsection{Chamada de funções}
	Quando uma função é chamada, seus parâmetros são empilhadas e posteriormente o endereço
	do retorno. Isso fica a encargo da função que faz a chamada.
	Para completar o \textsl{frame}, aquela que é chamada, empilha o endereço do frame da função chamadora
	(EBP) e posteriormente aloca na pilha o espaço correspondente a suas variáveis locais.
	É importante ressaltar que, caso o endereço de retorno, empilhado por quem chama, seja alterado,
	o fluxo de execução é mudado. Pois é justamente este o princípio do \textsl{buffer overflow}.

	\section{Funcionamento mais detalhado do Heap}
	A porção de memória correspondente ao heap possibilita ao programador alocar dinamicamente memória
	que fica disponível durante toda a execução para qualquer chamada de função. Diferentemente
	da memória alocada no stack - que é perdida quando a função retorna.

	\section{Registradores de controle}
	Uma parte fundamental da arquitetura que deve ser mencionada são os registradores que possuem
	relação direta com o gerenciamento da memória.
	Talvez o mais importante (na arquitetura base do estudo IA32) seja o EIP(Extended Instruction Pointer).
	Ele indica o endereço da próxima instrução. Sobrescrevê-lo equivale obter o controle
	do fluxo de um processo.
	Além dele, destacamos EBP(Extended Base Pointer) e ESP(Extended Stack Pointer).
	ESP indica o endereço do último valor inserido na pilha.
	O EBP indica o início da pilha para aquela chamada de função. É usado para referenciar variáveis
	locais da função.




\chapter{Classificação de vulnerabilidades}
\label{chap:classificacao}

	A classificação de vulnerabilidades representa enorme desafio.
	Nos dias de hoje, não existe nenhum padrão aceito globalmente para essa tarefa.
	Ainda assim, já houve vários avanços na área. 
	Existem padrões para enumerar e catalogar vulnerabilidades, bem como propostas
	que podem criar bases para uma classificação que venha a ser aceita pela comunidade.
	Métricas, relativas	à gravidade e ao impacto, também estão disponíveis
	e são empregadas no auxílio às instituições nas tomadas	de decisões.

	
	No trabalho de Seacord e Householder, \cite{Seacord2005}, temos os fatores que motivam a
	busca pela organização das vulnerabilidades em classes:
	\begin{itemize}
		\item{O entendimento das ameaças que elas representam.}
		\item{Correlacionamento de incidentes, de \textsl{exploits} e de artefatos.}
		\item{Avaliação da efetividade das ações de defesa.}
		\item{Descoberta de tendências de vulnerabilidades.}
	\end{itemize}

	
	Vemos, portanto, que a taxonomia\footnote{Ciência da classificação.} das vulnerabilidades
	pode trazer uma série de benefícios para seu entendimento, tratamento e prevenção.
	Nesse capítulo, nosso intuito é abordar a dificuldade nesse processo e apresentar
	os avanços já obtidos nesse sentido.  


	\section{A dificuldade em classificar; estágio já alcançado: enumeração}
		Antes de entrarmos no mérito das vulnerabilidades, é preciso definir
		com precisão dois termos que utilizaremos por todo o capítulo: classificar e enumerar.
		Como veremos, a taxonomia é mais custosa que a enumeração.

		\subsection{Classificar}
			\label{subsec:classificar}
			Como podemos encontrar em \cite{Holanda1975}, classificar implica "distribuir em classes e/ou grupos
			segundo um sistema". Logo, para a classificação, é preciso haver uma metodologia que possa
			separar os itens em estudo em diferentes grupos. A ciência que estuda esse processo
			é chamada taxonomia. Ela é guiada, conforme \cite{Gregio2005_1}, pelos princípios taxonômicos.
			São eles:
			\begin{description}
				\item[Exclusão mútua]
					Um item não podem ser categorizado simultaneamente em dois grupos.
				\item[Exaustividade]
					Os grupos, unidos, incluem todas as possibilidades.
				\item[Repetibilidade]
					Diferentes pessoas extraindo a mesma característica do objeto devem concordar com
					o valor observado.
				\item[Aceitabilidade]
					Os critérios devem ser lógicos e intuitivos para serem aceitos pela comunidade.
				\item[Utilidade]
					A classificação pode ser utilizada na obtenção de conhecimento na área de pesquisa.
			\end{description}

			
			Vemos que os critérios para a taxonomia são exigentes e pressupõem uma metodologia
			cuidadosamente gerada para atendê-los. 

		\subsection{Enumerar}
			A enumeração é um processo semelhante a 
			"indicar por números; relacionar metodicamente"; como encontramos
			em \cite{Holanda1975}.
			Trata-se, portanto, de algo muito mais simples que a classificação.
			Mesmo sendo mais simples, é extremamente importante pois permite
			que os itens enumerados sejam facilmente apontados e diferenciados entre si.
			
			
			Sem um procedimento de enumeração dos objetos de estudo, adotado de comum acordo,
			não é possível que duas partes se comuniquem sem risco de cometerem enganos. 
			Quem garante que estão tratando exatamente da mesma coisa naquele momento?
			Logo a enumeração é essencial para o devido entendimento sobre os objetos
			de estudo.

		\subsection{Da enumeração à classificação}
			No trabalho de Mann, \cite{Mann1999}, há um excelente paralelo entre a
			questão abordada nesse capítulo e o advento da tabela 
			periódica\footnote{Dispõe sistematicamente os elementos de acordo com suas propriedades permitindo
			uma análise multidimensional.} na Química. 
			A organização dos elementos da forma como conhecemos hoje na tabela periódica
			foi um processo longo que culminou com as ideias de Dimitri Mendeleev.
			Outros químicos que o precederam foram responsáveis pela identificação e
			listagem dos elementos. Isso possibilitou um melhor estudo e uma maior
			troca de informação precisa entre os pesquisadores.


			Segundo Mann, a tabela periódica só pode ser efetivamente criada graças
			aos esforços daqueles que enumeraram os elementos de forma mais simples
			antes de Mendeleev. O trabalho deles permitiu a interoperabilidade
			necessária para o surgimento da tabela periódica.
			Da mesma forma, nos anos antecedentes a 2000, a comunidade que estudava
			e acompanhava as vulnerabilidades estava num patamar semelhante àqueles
			que precederam Mendeleev. Ou seja, sequer havia uma enumeração mais
			amplamente aceita e reconhecida das vulnerabilidades que permitisse
			avanços suficientes para uma taxonomia.

			
			Citamos o ano de 2000 como parâmetro, pois nessa época, 1999, surgiria um projeto
			que se tornaria referência para a criação de uma padronização da enumeração
			de vulnerabilidades. Não seria ainda um evento comparável à criação da
			tabela periódica para Química (pois não trouxe a taxonomia) 
			mas certamente lançaria as bases para
			a interoperabilidade exigida para estudos mais aprofundados na área.
			Estamos falando da criação do 
			CVE(Common Vulnerabilities and Exposures)\footnote{http://cve.mitre.org}\footnote{Na época
			de sua criação era originalmente conhecido por Common Vulnerabilities Enumeration - vide
			\cite{Meunier2006} pg. 9.}
			pelo MITRE. A seção \ref{sec:cve} traz mais detalhes.


			Podemos dizer, portanto, que atualmente, embora não tenhamos uma taxonomia
			amplamente aceita pela comunidade, já foi atingido o estágio de enumeração.
			Projetos como o CVE podem ser considerados como marcos dessa etapa.
			A seguir, iremos abordar em mais detalhes o surgimento e o funcionamento dele.
			Isso nos possibilitará compreender melhor a complexidade da classificação
			das vulnerabilidades bem como irá facilitar o entendimento dos capítulos
			seguintes que abordam \textsl{exploits}.
			

	\section{CVE}
	\label{sec:cve}

		\subsection{Surgimento e objetivos}
			Para deixar mais nítida a dificuldade de interoperabilidade
			das organizações no que se refere a ameaças de segurança na época
			que antecede o CVE, segue abaixo uma tabela, extraída de \cite{Martin2001}.
			A tabela \ref{tab:torre_babel}, mostra
			como diferentes organizações se referiam à mesma vulnerabilidade
			em 1998. Trata-se de um verdadeira Torre de Babel.

			\begin{table}
				\begin{tabular}{|l|c|c|c|c|}
					\hline
						\textbf{Organização} & \textbf{Como se referia à vulnerabilidade}\\
					\hline
						CERT\footnotemark[1] & CA-96.06.cgi\_example\_code\\
					\hline
						Cisco Systems\footnotemark[2] & http - cgi-phf\\
					\hline
						DARPA & 0x00000025 = http PHF attack\\	
					\hline
						IBM ERS & ERS-SVA-E01-1996:002.1\\	
					\hline
						Security Focus\footnotemark[3] & \#629 - phf Remote Command Execution Vulnerability\\	
					\hline
				\end{tabular}
				\footnotetext[1]{site}
				\footnotetext[2]{site}
				\footnotetext[3]{site}
				\caption{Uma vulnerabilidade: diversos nomes e nenhum entendimento}\label{tab:torre_babel}
			\end{table}
			

			O CVE, como dito anteriormente, surge em 1999 e seu maior objetivo, como podemos
			ler em sua FAQ, \cite{CVE2010}, é tornar mais fácil
			o compartilhamento de informações sobre vulnerabilidades utilizando uma enumeração comum.
			Essa enumeração é realizada através da manutenção de uma lista na qual, conforme
			encontramos em \cite{Santos2004}, valem os seguintes princípios:
			\begin{itemize}
				\item{Atribuição de um nome padronizado e único a cada vulnerabilidade.}
				\item{Independência das diferentes perspectivas em que a vulnerabilidade ocorre.}
				\item{Abertura total voltada ao compartilhamento pleno das informações.}
			\end{itemize}


			Segundo a própria organização, vide \cite{CVE2010}, o CVE não possui um objetivo 
			inicial de conter alguma espécie de taxonomia. Essa é considerada uma área de pesquisa
			ainda em desenvolvimento. É esperado que, com o auxílio prestado pela
			catalogação das vulnerabilidades já constitua um importante passo para que
			isso ocorra.

			\begin{figure}
				\begin{center}
					\includegraphics[width=0.65\textwidth]{vulnerabilidades_CVE.jpg}
					\caption{Vulnerabilidades registradas no CVE ao longo do tempo - retirada 
							de \cite{Florian2009}}
					\label{fig:vulnerabilidades_CVE}
				\end{center}
			\end{figure}

			
			Na figura \ref{fig:vulnerabilidades_CVE}, podemos visualizar um histórico
			da quantidade de vulnerabilidades adicionadas. Nos últimos anos podemos
			perceber que os incidentes registrados ficam na média de 7000.
			Isso mostra relevância que o projeto do CVE alcançou.
			
			 
		
		\subsection{Funcionamento}
			O CVE é formado por uma junta de especialistas em segurança dos meios acadêmico, comercial e
			governamental. Eles são responsáveis por analisar e definir o que será feito dos reports passados
			pela comunidade - se eles devem ou não se integrar àqueles já pertencentes à lista.
			Cabe a eles definir nome, descrição e referências para cada nova ameaça.


			Esse processo inicia quando uma vulnerabilidade é reportada.
			Ela assume um CAN(Candidate Number), número de candidata.
			Até que ela seja adicionada à lista, ele permanece com um CAN que
			a identificará. Apenas após o devido estudo e aprovação do caso pela junta
			responsável, é que ela assume um identificador CVE.

			
			Os identificadores CVE são definidos conforme o padrão: CVE-2010-0021.
			Onde, separados por '-', há 3 partes. A primeira é fixa: CVE.
			A segunda refere-se ao ano de surgimento; enquanto a terceira
			indica o número sequencial daquela vulnerabilidade entre todas
			aquelas que foram adicionadas naquele ano. Logo, no exemplo fornecido,
			essa seria a vigésima primeira de 2010.
		
			
			Uma vez integrada, a vulnerabilidade passa a estar publicamente disponível.
			Essa abertura pode servir de auxílio aos atacantes - pois informações
			sobre possíveis furos de segurança são sempre bem vindas a eles.
			Porém, conforme podemos verificar na FAQ do CVE, \cite{CVE2010}, há uma série
			de motivos pelos quais a disponibilidade desses dados supera o risco
			oferecido pela exposição. São eles:
			\begin{itemize}
				\item{O CVE está restrito a publicar vulnerabilidades já conhecidas.}
				\item{Por diversas razões, a comunidade de segurança de informação
					sofre mais para compartilhar dados sobre as ameças
					que os atacantes.}
				\item{É muito mais custo a uma organização proteger toda sua rede
					contra as ameças que a um atacante descobrir e explorar uma delas para
					comprometer alguma das redes.}
			\end{itemize}
			
			

		
		
		
	\section{Propostas taxonômicas}
		Como foi referenciado anteriormente, não há padrão amplamente aceito
		para a taxonomia de vulnerabilidades.
		Mesmo assim, há diversas propostas que buscam resolver essa questão em aberto.
		
		
		O trabalho de \cite{Gregio2005} faz um levantamento amplo de algumas
		dessas alternativas. Mas, como veremos, nenhuma delas atinge os objetivos
		de uma taxonomia plena - apresentados na seção \ref{subsec:classificar}.
		Assim, buscaremos discutir as ideias para as metodologias
		de classificação com o intuito de apontar suas vantagens e fraquezas.

	\section{Métricas para vulnerabilidades}



\chapter{\textsl{exploits}}
\label{chap:exploits}
	No presente capítulo, será feita uma breve análise sobre \textsl{exploits}.
	É importante salientar que, apenas conhecendo as técnicas usadas pelos atacantes
	torna-se possível criar defesas efetivas contra elas.
	Portanto, o estudo dessa matéria não constitui, de forma alguma, uma apologia
	ao ataque. Essa questão é muito bem abordada na parte I de \cite{Harris2008}; deixando
	claro que o conhecimento é uma arma importantíssima para aqueles que buscam
	uma melhoria na segurança do software.
	

	Como ponto de partida, será aprofundado o conceito de \textsl{exploit}. De forma
	a mostrar sua amplitude e sua intrínseca relação com as vulnerabilidades. A seguir,
	serão explicadas algumas técnicas que são representativas para uma visão ampla do assunto. 
	Na sequência, serão abordados princípios básicos de programação que visam prevenir a aplicação
	de \textsl{exploits} no software - combatem as falhas na origem e pontos de apoio usados
	pelas técnicas dos atacantes. Por fim, serão apresentadas algumas das proteções já existentes
	para barrar os ataques; em certos casos, também serão mostradas as formas de escape que
	os atacantes já desenvolveram como reação. 
	

	Como não será detalhada nenhuma técnica em particular nesse capítulo, para se obter um exemplo
	mais aprofundado de \textsl{exploit}, o \textsl{NULL pointer exploit} será o tema do capítulo seguinte.
	Assim, após um acompanhamento mais amplo do tema, será possível compreender melhor um caso específico.

	\section{Definição}
		Conforme foi tratado na Seção \ref{sec:vuln_exploit}, o \textsl{exploit} é um conjunto de passos,
		muitas vezes materializado em um programa, capaz de tirar proveito de uma vulnerabilidade.
		Para muitos, entretanto, \textsl{exploit} é sinônimo de um código em C escrito por um \textsl{hacker}
		que tem o potencial de atacar um sistema. Essa visão, todavia, é muito limitada.
		Assim como existem diversos tipos de vulnerabilidades, há muitos meios de tirar vantagem
		delas. Por vezes, basta conhecer uma série de passos, como cliques na interface
		da aplicação alvo, para para explorar uma falha.

		
		Em \cite{Hoglund2004}, encontramos a seguinte lista de possíveis consequências
		para um \textsl{exploit} bem sucedido:
		\begin{itemize}
			\item{Parada parcial ou completa do sistema(DoS);}
			\item{Exposição de dados confidenciais;}
			\item{Escalada de privilégios;}
			\item{Execução de código injetado pelo atacante;}
		\end{itemize}
		Logo, ao explorar uma vulnerabilidade, podem ser gerados impactos na integridade, na confidencialidade
		ou na disponibilidade de um sistema.


		De modo geral, o grande objetivo de um atacante é conseguir executar código arbitrário
		em seu alvo. Isso, porém, nem sempre é possível. Cada vulnerabilidade, conforme analisado
		no capítulo anterior, determina um universo de possibilidades para um \textsl{exploit} que a ataque.


		Uma interessante forma de entender os \textsl{exploits}, sob a ótica de sua operação,
		está na separação deles em \textsl{\textbf{control-data}} e \textsl{\textbf{non-control-data}}.
		Conforme \cite{Chen2005}, ataques do tipo \textsl{control-data} são aqueles que
		alteram dados de controle do programa alvo (como endereço de retorno da função ou ponteiros)
		para executar código injetado ou desviar para outras bibliotecas. Os do tipo
		\textsl{non-control-data}, em contraponto, são aqueles que não alteram nenhum
		dado de controle do programa e não desviam seu fluxo de execução, mas conseguem
		alguma vantagem para o atacante - como autenticação ilegítima, elevação de privilégio, 
		leitura de dados confidenciais, etc. Ao apresentar os tipos de \textsl{exploits},
		será feito uso desse critério de classificação. Normalmente, os ataques que alteram
		estruturas de controle são aqueles que possibilitam execução arbitrária de código, enquanto
		os demais usam o próprio código da aplicação, explorando alguma falha de lógica ou de
		verificação.
		

	\section{Tipos}
		Nessa Seção, será feita uma breve explicação sobre alguns tipos de \textsl{exploits}
		existentes. Isso para que o leitor possa ter uma noção geral sobre as técnicas usadas
		pelos atacantes para explorar as vulnerabilidades no software.
		Não é, de forma alguma, uma lista exaustiva, mas contém muitos exemplos significativos.
		
		
		Abaixo, lista dos tipos abordados:
		\begin{itemize}
		 	\item{\textsl{Buffer overflow};}
			\item{\textsl{Heap overflow};}
			\item{Injeção de SQL;}
			\item{XSS(\textsl{Cross Site Scripting});}
		\end{itemize}

		\subsection{\textsl{Buffer Overflow}}
		\label{subsec:buffer_overflow}
			Um dos tipos mais bem conhecidos e um dos mais explorados. Tem um impacto
			enorme pois possibilita ao atacante a execução de código arbitrário no sistema atacado.
			O famoso artigo \textsl{Smashing the Stack for Fun and Profit} de 1996, por Aleph One,
			foi o primeiro a tratar em detalhes dessa técnica. Mas conforme, \cite{Anley2007},
			essa estratégia já vinha sendo aplicada com sucesso por mais de 20 anos antes da publicação
			do artigo de Aleph One.


			Ocorre quando a aplicação guarda dados a serem lidos dos usuário(ou de qualquer
			fonte externa) em um \textsl{buffer} alocado na pilha sem verificar
			se o que foi fornecido está dentro do limite aceitável(tamanho do \textsl{buffer}.
			Isso acaba resultando na grave falha que será apresentada abaixo.
			
			Conforme explicado na Seção \ref{sec:gerencia_memoria}, no \textsl{stack frame}
			existem valores que controlam o fluxo de execução de uma aplicação.
			Dentre eles, está o valor de retorno de uma rotina. Qualquer chamada de função
			coloca na pilha o endereço para o qual ela deve retornar após seu fim. Se esse
			valor for alterado, é possível mudar o fluxo da aplicação - fazendo
			com que ele seja desviado para outro ponto qualquer.
			

			Essa técnica tira proveito desse fato. Caso a aplicação possua alguma
			falha que permita que o usuário forneça dados maiores que o espaço
			alocado na pilha para armazená-los, o excedente acaba sobrescrevendo o
			endereço de retorno da função.
	

			É o exemplo claro de ataque \textsl{control-data}. A mudança em um dado
			de controle do \textsl{stack frame} permite a colocação de um endereço
			forjado pelo atacante para mudar o fluxo de execução do programa atacado.
			
			
			Na versão "clássica desse \textsl{exploit}, o atacante fornece
			código executável, \textsl{shellcode}, que vai além do \textsl{buffer} criado para
			armazená-lo. No final dos dados enviados, também é inserido o endereço de início
			do \textsl{buffer}, que agora contém o código do atacante, para substituir no
			\textsl{stack frame} o valor de retorno da função. Assim, no retorno o fluxo
			é desviado para o \textsl{buffer} com o \textsl{shellcode}. A figura 
			\ref{fig:pilha_buffer_overflow} mostra uma visão simplificada da pilha
			antes e depois do ataque. 

			\begin{figure}
				\begin{center}
					\includegraphics[width=0.90\textwidth]{pilha_buffer_overflow.jpg}
					\caption{Esquema da pilha no \textsl{buffer overflow}. Fonte: \cite{Martins2009}.}
					\label{fig:pilha_buffer_overflow}
				\end{center}
			\end{figure}

		\subsection{\textsl{Heap Overflow}}
			Semelhante ao \textsl{buffer overflow} quanto à falha que o provoca.
			Diferencia-se, entretanto, pelo fato do \textsl{buffer} a sofrer o \textsl{overflow}
			estar no heap e não na pilha. Pode ser de muito mais complexa execução que
			muitos outras técnicas - isso porque, conforme veremos, não possui o caráter
			mais genérico que seu equivalente para a pilha.
			
			
			Está diretamente relacionado à implementação feita para manejar o heap no sistema
			afetado. Isso geralmente é atribuição da biblioteca C. Logo, além da vulnerabilidade
			de \textsl{overflow}, deve estar presente uma versão de biblioteca C que faça alguma
			gerência incorreta do heap para que um ataque desse tipo seja possível.

			
			Assim, um \textsl{exploit} de \textsl{heap overflow} é totalmente focado
			em uma determinada versão de biblioteca C de um sistema, pois normalmente as aplicações
			não fazem sua própria gerência do heap. Na Seção \ref{sec:funcionamento_heap}, há
			uma explicação do funcionamento do heap que auxilia na compreensão desse tipo de ataque.
			

			Para ilustrarmos melhor esse tema, iremos focar em um sistema específico
			para mostrar a sistemática e a potencialidade de um \textsl{heap overflow}.
			Será a implementação de gerência do heap do Linux originalmente escrito por Doug Lee.
			A tarefa de controle da memória dinâmica é extremamente complexa e desafiadora, pois
			está condicionada a otimização temporal e espacial. Como muitas aplicações fazem
			uso intensivo de chamadas a malloc, free e mmap - todas para controle do heap - é preciso 
			um enorme cuidado para que os recursos de CPU e memória sejam bem utilizados de forma
			a não prejudicar o desempenho da aplicação e do sistema como um todo.


			Para manter controle do heap, nos blocos alocados e fornecidos às aplicações,
			são também postos dados de manutenção. São meta informações que visam auxiliar
			na administração dos blocos de memória. Assim, por exemplo, ao alocarmos uma porção
			de memória utilizando malloc, escondido no bloco, teremos dados que a biblioteca
			mantém.	Para a referida versão da biblioteca C do Linux, havia uma falha na qual, 
			uma vez que os metadados dos blocos fossem alterados (via \textsl{overflow}) de uma
			determinada forma, o atacante poderia conseguir uma escrita arbitrária e um endereço arbitrário.
			Conforme já abordado anteriormente, uma falha dessa magnitude implica a possibilidade
			de alteração do fluxo da aplicação caso seja sobrescrita alguma estrutura de dados de controle.
			Uma alternativa seria um ponteiro para um função - pois uma vez sobrescrito, bastaria
			que ele passasse a apontar para o código injetado pelo atacante.


			Como a intenção desse capítulo é fornecer uma visão geral, não será detalhada
			a construção do ataque. Apenas serão esboçados os passos que são seguidos pelo atacante
			para desenvolver o \textsl{exploit}.
			
			Em, \cite{Anley2007}, 
				
		
		\subsection{Injeção de SQL}
			Diferentemente dos \textsl{exploits} anteriores, não se trata de um erro de corrupção
			de memória. Serve como boa forma de contraponto para mostrar que um sistema
			pode ter sua confidencialidade e integridade afetados de outra forma.
			Ocorre na camada de banco de dados de uma aplicação em virtude de uma filtragem inadequada
			dos dados usados para gerar \textsl{queries} SQL.			

			
			Ainda que seja uma classe muito diferente, quando comparado aos 2 tipos 
			descritos anteriormente, cabe destacar que seria identificado como 
			\textsl{non-control-data}. Não é necessária nenhuma alteração no fluxo do programa
			explorado.

			
			Sua potencialidade é enorme. Como implica a possibilidade do atacante
			injetar \textsl{queries} no banco de dados do sistema alvo, significa
			dizer que ele terá todos os privilégios de acesso que a aplicação possuir.
			Pode ser possível expor informações sigilosas, alterá-las ou mesmo destruir
			toda a base de dados.


			A técnica desse \textsl{exploit}, portanto, consiste em utilizar os comandos
			SQL previstos na aplicação para executar ações de interesse do atacante - expondo ou
			alterando dados de forma não prevista.

		\subsection{\textsl{XSS}(\textsl{Cross Site Scripting})}
			Um dos ataques mais difundidos na web. Conforme \cite{Dhanjani2009},
			é o meio mais comum de ataques a clientes web - constituindo poderosa arma
			contra a rede interna das corporações. Trata-se de um ataque voltado para o lado
			do cliente - diferentemente daqueles expostos anteriormente - que buscam explorar o servidor.


			Seu funcionamento básico se dá através da injeção de código malicioso por atacantes
			em páginas web. Esse código acaba sendo executado por clientes sem seu conhecimento.
			Isso possibilita aos atacantes obter acesso a dados restritos mantidos pelos clientes
			nos \textsl{browsers}. Uma das possíveis implicações é o roubo de sessões web - tornando
			o atacante capaz de acessar o servidor, indistintamente, com os mesmos privilégios
			do usuário legítimo.
			

			É um problema semelhante à injeção de SQL - já que a validação imprópria(ou mesmo inexistente)
			permite	que código malicioso seja processado pelo servidor e posto no conteúdo de suas páginas
			para ser entregue a outros usuários. Isso confirma, novamente, a premente necessidade de validação
			de todo e qualquer dado de entrada em uma aplicação. 

			
			Uma das técnicas mais utilizadas para roubo de sessões, descrita em \cite{Dhanjani2009}, é 
			a injeção de código Javascript no servidor para repassar ao atacante
			todos os dados da sessão do cliente que acesse a página. Para isso, o atacante mantém um servidor
			que é acionado toda vez que um cliente processa o script que ele injetou no servidor vulnerável.
			Esse script executado no cliente, vítima, fornece ao servidor do atacante toda informação
			necessária para que seja possível assumir a identidade dela.
			
			
			Outro possível ataque, também apresentado em \cite{Dhanjani2009} é o roubo de senhas armazenadas
			nos \textsl{browsers} dos clientes.	Isso ocorrer quando alguma vítima utiliza o recurso
			de armazenamento de senhas. Muito embora isso constitua uma comodidade, uma vez que o servidor
			esteja vulnerável a XSS, os atacantes podem, através de script forjado para fingir
			um login injetado no servidor, recuperar as senhas armazenadas \textsl{browser}. 


			Para aprofundamento nas técnicas de XSS e para maior conhecimento nas formas de prevenção,
			é aconselhável a leitura do capítulo 2 de \cite{Dhanjani2009}. Há riqueza de exemplos e
			derivações do XSS que constituem nova geração dessa forma de ataque.
			

	\section{Prevenção de ataques}
		Para prevenir as indesejáveis consequências dos \textsl{exploits} apresentados
		anteriormente, mas não se restringindo a eles, serão discutidos princípios básicos
		para o desenvolvimento do software. São meios de trazer maiores garantias contra os ataques
		na origem. Será demonstrado que a validação dos dados usados pelas aplicações, bem como
		o uso de ferramentas de análise de código e de testes são exigências que não podem
		ser desconsideradas. 

		
		\subsection{Validação de dados de entrada}
			Um dos pontos primordiais para a defesa contra os ataques é a validação dos dados de entrada.
			Sendo esse procedimento capaz de deter uma série de ameaças. Uma aplicação que não verifique
			devidamente os dados que lhe são fornecidos é séria candidata a ser explorada. Não é possível
			confiar em nada que advém de qualquer ponto externo ao sistema. Conforme visto anteriormente,
			ataques como o de \textsl{buffer overflow} ou de \textsl{heap overflow} estão diretamente
			ligados a uma validação incorreta(ou mesmo ausente) de dados de entrada. O mesmo ocorrendo
			para injeção de SQL ou XSS.


			Para que essa prática seja bem aplicada, é essencial que sejam levantados todos os vetores
			de entrada de uma aplicação. Por vezes, alguns deles podem ser esquecidos. No ambiente UNIX,
			por exemplo, variáveis de ambiente também devem ser consideradas dados de entrada. Entretanto,
			nem sempre são devidamente validadas. Nesse aspecto, toda uma preocupação com a entrada
			do sistema pode ser perdida se restar apenas um ponto não verificado. Por isso a exigência
			de uma avaliação dos pontos que devem ser protegidos.


			Para ilustrar ainda melhor, podemos tomar como exemplo um sistema que faça uso de 
			DNS reverso\footnote{Processo de descoberta do nome associado a um dado IP.}. Se, para um
			dado IP, não for validado o nome retornado pelo DNS reverso, um atacante pode, uma vez que
			tenha comprometido parte da rede, forçar a aplicação a utilizar dados impróprios.
			Se a aplicação do exemplo usar diretamente o resultado, ela corre sérios riscos de sofrer
			algum tipo de exploração - como um \textsl{buffer overflow}.


			É, fundamental, portanto, que os pontos de entrada sejam identificados e sejam
			definidas formas de validação. Em \cite{Secure2006}, anexo B, há detalhes sobre
			esse tópico - definindo diretivas para a validação.
			
		\subsection{Ferramentas de análise estática e auditoria de código}
			Uma das melhores formas de prevenção a ataques é auditar o código. 
			A busca por falhas não precisa ser um procedimento manual; há uma série de ferramentas,
			algumas delas sofisticadas e focadas nessa tarefa, que podem facilitar muito a vida
			dos desenvolvedores.
			Nessa Seção, iremos abordar essa estratégia na busca por problemas 
			que possam ser eliminados já na fase de desenvolvimento - procurando
			deixar o mínimo possível de brechas para os atacantes.
			
			
			Conforme \cite{Ari2008}, a auditoria de código cai na categoria de teste
			estrutural caixa-branca. Isso porque parte do código fonte para desempenhar sua tarefa.
			O mesmo autor também destaca que esse processo,
			assim como os testes fuzzing(vide capítulo \ref{chap:fuzzing}), não é
			\textsl{capaz de comprovadamente encontrar todos os \textsl{bugs} ou erros possíveis}. 
			Ainda assim, ele recomenda fortemente seu uso em complementação a outras técnicas
			de testes(como o fuzzing ou outros tipos de teste caixa-preta).


			Muito embora o uso de ferramentas estáticas não possa substituir um auditor experiente,
			conforme ressalta \cite{Anley2007}, elas podem servir de base para a tarefa.
			Em sua maioria, elas possuem uma base de dados de padrões de código perigoso.
			É o caso do uso da função strcpy para a cópia de strings. Uma linha de que contenha
			esse tipo de chamada será encontrada e reportada como problema a ser tratado - dado
			o risco que ela representa.
			Como exemplos de ferramentas para auxílio na busca por falhas no software, temos:
			\begin{description}
				\item[Splint]{Faz análise de falhas de código C. Segundo \cite{Anley2007}, é capaz
					de realizar algumas verificações bem complexas.}
				\item[RATS]{Busca por falhas já bem conhecidas em linguagens com C, C++, Perl e Python.
					Não possui a mesma profundidade nas verificações que Splint, mas é uma ótima
					forma de garantir a ausência de problemas já superados.}
				\item[Flawfinder]{Semelhante a RATS. Ambas surgem simultaneamente e cobrem uma mesma
					gama de falhas em suas verificações. Encontrada em: http://www.dwheeler.com/flawfinder/.}
			\end{description}
			

		\subsection{Testes}
			Mesmo que tenham sido tomadas as devidas precauções e que o código tenha sido auditado,
			ainda podem existir problemas desconhecidos no software. Falhas exploráveis que, muitas vezes,
			surgem somente quando a aplicação é testada - principalmente contra problemas de segurança.
			

		
	
	\section{Proteções e contra-proteções}
	\label{sec:exploit_protection}
		Existem diversas proteções para impedir um \textsl{exploit}.
		São recursos dos compiladores, das bibliotecas, do hardware e dos sistemas operacionais
		que servem de contra ponto às mais variadas técnicas que os atacantes já criaram.
		Seu principal objetivo é resguardar os sistemas mesmo que os desenvolvedores
		não tenham seguido as recomendações de segurança. De forma que, mesmo na presença de uma
		vulnerabilidade, um ataque não seja possível ou seus efeitos sejam minimizados ao máximo.
		
		
		Conforme é possível encontrar em \cite{Anley2007}, destacamos os seguintes mecanismos
		de proteção:
		\begin{enumerate}
			\item{Pilha não executável;}
			\item{W \^\ X(permissão de escrita ou de execução - nunca ambas);}
			\item{Canário para pilha;}
			\item{Reordenamento das variáveis na pilha;}
			\item{ASLR - Randomização do espaço de endereços;}
		\end{enumerate}

		
		A seguir, cada uma será explicada em seus aspectos fundamentais.
		\subsection{Pilha não executável}
			A primeira, pilha não executável, é uma reação natural a um dos ataques mais comuns:
			o \textsl{buffer overflow}. Há registros de propostas de pilha não executável desde 1996 -
			conforme \cite{Anley2007}(pg. 376). O \textsl{exploit} clássico sendo baseado na cópia
			de \textsl{shell code} para o buffer e posterior execução dele ficaria impraticável.
			Mas não demorou muito para os atacantes reagirem. Surgiram novas técnicas que funcionam
			mesmo quando não é possível executar o código injetado na pilha.
			Sua estratégia básica era: a partir do controle do \textsl{stack frame}, criar uma chamada
			válida para biblioteca C ou chamadas de sistema. Inicialmente, ela foi denominada \textbf{return-into-libc}.
			
			
			Essa nova técnica de \textsl{exploit} abriria caminho para uma série de outras.
		

		\subsection{W\^\ X}
			Impedir que memória com proteção de escrita seja executável e, bloquear a escrita
			para aquela que é executável é uma das melhores formas de proteção.
			Ataca justamente um princípio fundamental da maioria das técnicas de ataque: injetar código
			(escrever) e executá-lo.	
			

			Embora essa técnica seja hoje em dia conhecida pelo batismo de Theo Raadt, desenvolvedor
			e líder do projeto do OpenBSD, ela tem sua origem na década de 1970. Em \cite{Anley2007},
			é mencionado que o sistema Multics teria sido um dos pioneiros a contar com esse tipo de
			proteção. Para facilitar essa estratégia de defesa na arquitetura x86, em 2003, a AMD
			criaria o \textbf{NX(Non-eXecutable)}. Um suporte no hardware que identificasse uma página de
			memória que não pudesse ser executada. O equivalente da Intel seria o \textsl{ED(Execute Disable)}.


			Mesmo sendo uma excelente forma de impedir ataques, isolada, essa defesa não é capaz
			suficiente. Algumas técnicas derivadas de \textsl{return-into-libc} são imunes.


		\subsection{Canário para a pilha}
			Outra forma de proteção para a pilha é colocação de um canário.
			Trata-se de um valor(normalmente de 32 bits) que é posto no \textsl{stack frame}
			para identificar se houve um \textsl{overflow} na pilha.
			A figura \ref{fig:canario} ilustra essa proteção. O canário é posto de forma
			a proteger o endereço de retorno. Ao término da chamada da função, ele é verificado
			e, caso não seja o valor esperado, a aplicação é terminada.


			Sua primeira implementação foi o  \textsl{StackGuard} em 1998, vindo a fazer
			parte do compilador GCC(GNU Compiler Collection) - posteriormente sendo
			substituído pelo SSP(\textsl{Stack-Smashing Protector}) \cite{Martins2009}.
			O SSP além de implementar proteção por canário, também atua reordenando
			as variáveis da pilha para aumentar a segurança - conforme explicado 
			na Seção \ref{subsec:reordenamento_pilha}.

			
			Atualmente é uma proteção padrão em quase todos os sistemais operacionais e certamente
			contribui muito para frear \textsl{exploits} de \textsl{buffer overflow}. Sua
			proteção é ainda maior quando combinada com o reordenamento da pilha.
			

			\begin{figure}
				\begin{center}
					\includegraphics[width=0.30\textwidth]{canario.jpg}
					\caption{\textsl{Stack frame} protegido por canário. Fonte: \cite{Furlan2005}.}
					\label{fig:canario}
				\end{center}
			\end{figure}

		\subsection{Reordenamento de variáveis na pilha}
		\label{subsec:reordenamento_pilha}
			É aplicada pelo SSP e complementa a proteção oferecida pelo canário.
			É uma barreira extra para que um \textsl{overflow} nos \textsl{buffers} - que só
			é detectado após o término da função - não seja usado para afetar outras variáveis.


			Seu objetivo é, conforme \cite{Martins2009}:
			"isolar os arrays que podem vir a vazar dados, para que seu estouro não
			afete as outras variáveis locais da função. Isso garante a integridade das variáveis
            automáticas no decorrer da função, e evita o seu possível uso para a injeção de \textsl{shellcode}".

			
			É baseada em um modelo ideal de pilha no qual as variáveis locais que não
			são \textsl{buffers} são melhor protegidas contra possíveis \textsl{overflows}.
			O modelo é melhor compreendido através da visualização da figura \ref{fig:pilha_ideal_ssp}.

			\begin{figure}
				\begin{center}
					\includegraphics[width=0.80\textwidth]{pilha_ideal_ssp.jpg}
					\caption{Modelo de pilha ideal para o SSP. Fonte: \cite{Martins2009}.}
					\label{fig:pilha_ideal_ssp}
				\end{center}
			\end{figure}


		\subsection{ASLR}
			O \textsl{\textbf{Address Space Layout Randomization}} implementa uma randomização
			dos endereços de forma a dificultar enormemente a vida dos atacantes.
			Bibliotecas e rotinas passam a ter endereços aleatórios e os saltos necessários
			para esses endereços ficam muito mais complexos de serem realizados.

			
			Conforme explicado anteriormente, vários \textsl{exploits} dependem de um conhecimento
			prévio dos endereços. Portanto, essa aleatoriedade é muito interessante como
			forma de proteção genérica. Sua fraqueza, porém, conforme \cite{Anley2007}, está no fato
			de bastar algum endereço fixo para que ela não tenha efeito algum. Mas nem sempre é necessário
			que haja algo fixo; há uma técnica chamada \textsl{\textbf{heap spraying}} que é capaz
			de driblar o ASLR. Ela injeta várias porções de código executável na aplicação alvo
			para que, mesmo desconhecendo um endereço preciso, a chance de que ele seja encontrado
			venha a ser muito maior.

			
			Há mais detalhes sobre \textsl{\textbf{heap spraying}} em \cite{Nozzle}. No referido trabalho,
			inclusive, é sugerido um verificador de \textsl{heap} que procura impedir que esse tipo
			de ataque sej. No referido trabalho,
			inclusive, é sugerido um verificador de \textsl{heap} que procura impedir que esse tipo
			de ataque seja aplicado.
			
			
			

		

		
			
	


\chapter{NULL pointer Exploit}
\label{chap:null_pointer_exploit}

	Dentre as várias técnicas de exploits existentes, uma que certamente merece
	destaque, é o NULL pointer exploit.
	Surgiu recentemente e é fruto da crescente dificuldade em aplicar técnicas
	que exploram vulnerabilidades de corrupção de memória.
	Em princípio, deferenciar um ponteiro nulo não é considerada
	uma vulnerabilidade explorável, mas, como veremos, isso nem sempre é verdade.

	
	O ano de 2009 chegou a ser considerado o ano do "kernel NULL pointer deference"
	em virtude da grande quantidade de falhas desse gênero encontradas no kernel do Linux.
	(citar http://cwe.mitre.org/top25/ para o ano de 2009)
	Vemos então que essa falha chegou a atingir notoriedade pelo grande impacto que causou.
	Nossa intenção é apresentar esse tipo de exploit, mostrar como pode ocorrer, ilustrando
	com exemplos e apontando formas de prevenção.
	
	
	
	\section{O que é um NULL pointer}
		O primeiro ponto a ser abordado é o NULL pointer.
		Na linguagem de programação C, podemos considerar um ponteiro como um valor inteiro
		que referencia uma posição de memória. Ou seja, trata-se de um valor que aponta
		para um determinado ponto no espaço de endereçamento. Quando um ponteiro é deferenciado,
		passamos a acessar o valor presente na posição de memória para o qual ele aponta.
		Ilustrando, segue pequeno trecho de código C.
		\begin{lstlisting}[label=pointer_example,caption=Ponteiro em C]
int val = 10;
int *pointer = &value;
/* pointer has the address of val */
/* *pointer returns the value stored in val */
		\end{lstlisting}
		

		No Linux, o arquivo stddef.h contém a definição de NULL, que por
		convenção, denomina um ponteiro com valor zero.
		Um ponteiro nulo, então, aponta para a posição zero de memória. 
		Como, em regra geral, os sistemas utilizam o esquema de memóra virtual,
		na prática, esse endereço zero deve ser considerado tão somente no espaço
		de endereçamento do processo em questão.
		Como normalmente ele não constitui um mapeamento válido, pois os processos não
		iniciam com aquela porção mapeada, os acessos a essa região implicam violação
		às regras do esquema de memória virtual. Erros desse gênero resultam em
		segmentation fault.   

		Logo, quando deferenciamos um ponteiro nulo, temos um acesso não previsto
		que configura uma violação 
		Por essa razão, que o acesso a um ponteiro com valor
		zero acaba sendo um erro na aplicação que resulta em sua finalização.
		
		Embora o foco do trabalho recaia sobre a arquitetura x86, é válido identificar a repercussão
		de uma acesso a posição zero de memória em outros casos.
		Existem arquiteturas nas quais o endereço zero já é mapeado inicialmente. 
		Podemos apontar o caso da ARM e da XScale; ambas para sistemas embarcados. 
		Nelas, o vetor de exceções se encontra nessa posição. Ele contém
		endereços para o tratamento de, por exemplo, interrupções de hardware e software.
		Naturalmente, um acesso a um ponteiro nulo nesses casos constitui enorme
		problema em termos de segurança.

		

	\section{Como funciona a técnica}

	\section{Exemplos}

	\subsection{Flash Player exploit}
	
	\subsection{Falhas no kernel do Linux}
		Existem diversas falhas no kernel do Linux relacionadas a NULL pointer.
		Desde problemas na inicialização de estruturas de dados até a erros na compilação.		
	
	\subsection{Pidgin}

	\section{Proteções}


\chapter{Fuzzing: detecção de vulnerabilidades}
\label{chap:fuzzing}

	\section{O que é fuzzing?}

	\section{Fuzzing aplicado ao teste de software}

	\section{Detectando vulnerabilidades}

	\section{Exemplos}



\chapter{Conclusão}
\label{chap:conclusao}
	Nesse trabalho foram abordados aspectos essenciais relacionados à segurança do software.
	Ao tratar de vulnerabilidades e técnicas de \textsl{exploits}, ele objetivou
	trazer ao leitor um contexto fundamental para um entendimento da área. Dada a relevância
	que o software atingiu nos dias de hoje, não é mais admissível que qualquer desenvolvimento
	sério desconsidere princípios de segurança. Sendo eles: as possíveis vulnerabilidades,
	as formas de ataque e, naturalmente, as formas de prevenção.


	No que se refere às vulnerabilidades, esse trabalho, ao tratar de sua classificação, pode identificar
	que esse tópico ainda não é pacífico no meio acadêmico ou industrial. Mesmo que tenham
	sidos feitos avanços, a comunidade carece de um padrão aceito uniformemente. Ficou nítido que a complexidade
	dessa tarefa é enorme. O próprio caráter multifacetado das vulnerabilidades explica um pouco essa barreira;
	elas podem ser analisados por diversos ângulos e estamos longe de encontrar uma visão unificadora
	que traga sentido a todas suas faces. Apenas assim seria alcançada uma taxonomia em sentido estrito.


	No campo dos \textsl{exploits}, pode ser visto que, com a evolução natural
	das técnicas de ataque e de defesa, a vida dos especialistas na área torna-se cada vez mais árdua.
	As formas mais simples de explorar vulnerabilidades já não são mais efetivas; seja porque as falhas
	que as tornam possíveis ficaram menos frequentes no desenvolvimento, seja porque proteções mais
	bem concebidas foram sendo habilitadas por padrão nos sistemas. Isso vai obrigando os atacantes
	a encontrarem novos métodos cada vez mais sofisticados que, naturalmente, vão exigindo conhecimento
	ainda mais específico. Em alguns momentos, porém, ainda será possível surgir alguma espécie
	de reviravolta - como na descoberta dos diversos erros de \textsl{NULL pointer} no kernel do Linux.
	Episódio que demonstrou a existência de uma série de falhas por vários anos em um dos sistemas
	mais utilizados - surpreendentemente, algumas delas de fácil exploração.


	Para a prevenção de problemas de segurança no software, uma das principais propostas apresentadas foi
	o uso do testes fuzzing. Sendo uma técnica extremamente eficiente e que já vem sendo usada
	por atacantes para a descoberta de problemas nos sistemas, esse trabalho buscou demonstrar seu valor
	e indicá-la como arma a ser utilizada pelos próprios desenvolvedores. Por que já não conceber, desde
	o princípio, um projeto considerando essa alternativa de teste se os atacantes certamente irão utilizá-la?
	Conforme foi visto, gigantes da área, como a Microsoft, já perceberam seu enorme valor e investem
	fortemente nela. É necessário, portanto, que, ao menos, consideremos essa possibilidade. Isso
	porque não é aceitável correr o risco de deixar apenas para os atacantes utilizarem e aperfeiçoarem
	uma técnica que possa desequilibrar em favor deles.



%
% referencias
%
\bibliographystyle{abnt}
\bibliography{bibliografia}

%
% apendix
%
\appendix

\chapter{Equações CVSS 2.0}
\label{chap:equacoes_cvss}


	\section{Equações do escore básico}
		\begin{equation}
			\label{eq:basica}
			EscoreBasico = ((0.6 * Impacto)+(0.4 * Explorabilidade) - 1.5)*algo
		\end{equation}

		\begin{equation}
			\label{eq:impacto}
			Impacto = 10.41 * (1-(1-ImactoConf) * (1-ImpactoInt) * (1-ImpactoDisp))
		\end{equation}

		\begin{equation}
			\label{eq:explorabilidade}
			Explorabilidade = 20 * VetorAcesso * ComplexidadeAcesso * NecessidadeAut
		\end{equation}

	\section{Equações do escore temporal}
		Utiliza o escore básico.
		\begin{equation}
			\label{eq:temporal}
			EscoreTemporal = EscoreBasico * FacExploracao * NivelRemed * ConfReport
		\end{equation}

	\section{Equações do escore ambiental}
	Utiliza o escore temporal.
		\begin{equation}
			\label{eq:ambiental}
			EscoreTemporal = EscoreBasico * FacExploracao * NivelRemed * ConfReport
		\end{equation}
	

	% Metricas básicas
	\begin{table}
		\begin{tabular}{|c|c|c|}
			\hline
			\multicolumn{3}{|c|}{ \textbf{Métricas básicas} } \\
			\hline
			\textbf{Métrica} & \textbf{Valor nominal} & \textbf{Valor numérico}\\
			\hline
			\multirow{3}{*}{Vetor de acesso} & local & 0.395 \\
			& rede adjacente & 0.646 \\
			& rede & 1.0 \\ 
			\hline
			\multirow{3}{*}{Complexidade de acesso} & alta & 0.35 \\
			& média & 0.61 \\
			& baixa & 0.71 \\
			\hline
			\multirow{3}{*}{Necessidade de autenticação} & várias & 0.45 \\
			& uma & 0.56\\
			& nenhuma & 0.704\\
			\hline
			\multirow{3}{*}{Impacto na confidencialidade} & nenhum & 0.0 \\
			& parcial & 0.275\\
			& completo & 0.660\\
			\hline
			\multirow{3}{*}{Impacto na integridade} & nenhum & 0.0 \\
			& parcial & 0.275\\
			& completo & 0.660\\
			\hline
			\multirow{3}{*}{Impacto na disponibilidade} & nenhum & 0.0 \\
			& parcial & 0.275\\
			& completo & 0.660\\
			\hline
		\end{tabular}
	\end{table}

	% Metricas temporais
	\begin{table}
		\begin{tabular}{|c|c|c|}
			\hline
			\multicolumn{3}{|c|}{ \textbf{Métricas temporais} } \\
			\hline
			\textbf{Métrica} & \textbf{Valor nominal} & \textbf{Valor numérico}\\
			\hline
			\multirow{5}{*}{Facilidade de exploração} & não comprovada & 0.85 \\
			& prova de conceito & 0.9 \\
			& funcional & 0.95 \\ 
			& alta & 1.0 \\
			& não definida & 1.0 \\ 
			\hline
			\multirow{5}{*}{Nível de remediação} & conserto definitivo  & 0.87 \\
			& conserto temporário & 0.90\\
			& \textsl{workaround} & 0.95\\
			& indisponível & 1.0\\
			& não definido & 1.0\\
			\hline
			\multirow{4}{*}{Confiabilidade no \textsl{report}} & não confirmada & 0.9 \\
			& não corroborada & 0.95\\
			& confirmada & 1.0\\
			& não disponível & 1.0\\
			\hline
		\end{tabular}
	\end{table}

	% Metricas ambientais
	\begin{table}
		\begin{tabular}{|c|c|c|}
			\hline
			\multicolumn{3}{|c|}{ \textbf{Métricas ambientais} } \\
			\hline
			\textbf{Métrica} & \textbf{Valor nominal} & \textbf{Valor numérico}\\
			\hline
			\multirow{6}{*}{Dano colateral potencial} & nenhum & 0.0\\
			& baixo        & 0.1 \\
			& baixo-médio  & 0.3 \\ 
			& médio-alto   & 0.4 \\
			& alto         & 0.5 \\ 
			& não definido & 0 \\ 
			\hline
			\multirow{4}{*}{Abundância de alvos} & nenhuma & 0 \\
			& baixa & 0.25\\
			& média & 0.75\\
			& alta  & 1.0\\
			& não definida  & 1.0\\
			\hline
			\multirow{4}{*}{Importância da confiabilidade} & baixa & 0.5 \\
			& média & 1.0\\
			& alta & 1.51\\
			& não definida & 1.0\\
			\hline
			\multirow{4}{*}{Importância da integridade} & baixa & 0.5 \\
			& média & 1.0\\
			& alta & 1.51\\
			& não definida & 1.0\\
			\hline
			\multirow{4}{*}{Importância da disponibilidade} & baixa & 0.5 \\
			& média & 1.0\\
			& alta & 1.51\\
			& não definida & 1.0\\
			\hline
		\end{tabular}
	\end{table}


\end{document}
