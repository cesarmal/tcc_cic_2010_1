
\chapter{Classificação de exploits}
\label{chap:classificacao}

	A classificação de vulnerabilidades representa enorme desafio.
	Nos dias de hoje, não existe nenhum padrão aceito globalmente para essa tarefa.
	Ainda assim, já houve vários avanços na área. 
	Existem padrões para enumerar e catalogar vulnerabilidades, bem como propostas
	que podem criar bases para uma classificação que venha a ser aceita pela comunidade.
	Métricas, relativas	à gravidade e ao impacto, também estão disponíveis
	e são empregadas no auxílio às instituições nas tomadas	de decisões.

	
	No trabalho de Seacord e Householder, \cite{Seacord2005}, temos os fatores que motivam a
	busca pela organização das vulnerabilidades em classes:
	\begin{itemize}
		\item{O entendimento das ameaças que elas representam.}
		\item{Correlacionamento de incidentes, de \textsl{exploits} e de artefatos.}
		\item{Avaliação da efetividade das ações de defesa.}
		\item{Descoberta de tendências de vulnerabilidades.}
	\end{itemize}

	
	Vemos, portanto, que a taxonomia\footnote{Ciência da classificação.} das vulnerabilidades
	pode trazer uma série de benefícios para seu entendimento, tratamento e prevenção.
	Nesse capítulo, nosso intuito é abordar a dificuldade nesse processo e apresentar
	os avanços já obtidos nesse sentido.  


	\section{A dificuldade em classificar; estágio já alcançado: enumeração}
		Antes de entrarmos no mérito das vulnerabilidades, é preciso definir
		com precisão dois termos que utilizaremos por todo o capítulo: classificar e enumerar.
		Como veremos, a taxonomia é mais custosa que a enumeração.

		\subsection{Classificar}
			Como podemos encontrar em \cite{Holanda1975}, classificar implica "distribuir em classes e/ou grupos
			segundo um sistema". Logo, para a classificação, é preciso haver uma metodologia que possa
			separar os itens em estudo em diferentes grupos. A ciência que estuda esse processo
			é chamada taxonomia. Ela é guiada, conforme \cite{Gregio2005_1}, pelos princípios taxonômicos.
			São eles:
			\begin{description}
				\item[Exclusão mútua]
					Um item não podem ser categorizado simultaneamente em dois grupos.
				\item[Exaustividade]
					Os grupos, unidos, incluem todas as possibilidades.
				\item[Repetibilidade]
					Diferentes pessoas extraindo a mesma característica do objeto devem concordar com
					o valor observado.
				\item[Aceitabilidade]
					Os critérios devem ser lógicos e intuitivos para serem aceitos pela comunidade.
				\item[Utilidade]
					A classificação pode ser utilizada na obtenção de conhecimento na área de pesquisa.
			\end{description}

			
			Vemos que os critérios para a taxonomia são exigentes e pressupõem uma metodologia
			cuidadosamente gerada para atendê-los. 

		\subsection{Enumerar}
			A enumeração é um processo semelhante a 
			"indicar por números; relacionar metodicamente" - como encontramos
			em \cite{Holanda1975}.
			Trata-se, portanto, de algo muito mais simples que a classificação.
			Mesmo sendo mais simples, é extremamente importante pois permite
			que os itens enumerados sejam facilmente apontados e diferenciados entre si.
			
			
			Sem um procedimento de enumeração dos objetos de estudo, adotado de comum acordo,
			não é possível que duas partes se comuniquem sem risco de cometerem enganos. 
			Quem garante que estão tratando exatamente da mesma coisa naquele momento?
			Logo a enumeração é essencial para o devido entendimento sobre os objetos
			de estudo.

		\subsection{Da enumeração à classificação}
			No trabalho de Mann, \cite{Mann1999}, há um excelente paralelo entre a
			questão abordada nesse capítulo e o advento da tabela 
			periódica\footnote{Dispõe sistematicamente os elementos de acordo com suas propriedades permitindo
			uma análise multidimensional.} na Química. 
			A organização dos elementos da forma como conhecemos hoje na tabela periódica
			foi um processo longo que culminou com as ideias de Dimitri Mendeleev.
			Outros químicos que o precederam foram responsáveis pela identificação e
			listagem dos elementos. Isso possibilitou um melhor estudo e uma maior
			troca de informação precisa entre os pesquisadores.


			Segundo Mann, a tabela periódica só pode ser efetivamente criada graças
			aos esforços daqueles que enumeraram os elementos de forma mais simples
			antes de Mendeleev. O trabalho deles permitiu a interoperabilidade
			necessária para o surgimento da tabela periódica.
			Da mesma forma, nos anos antecedentes a 2000, a comunidade que estudava
			e acompanhava as vulnerabilidades estava num patamar semelhante àqueles
			que precederam Mendeleev. Ou seja, sequer havia uma enumeração mais
			amplamente aceita e reconhecida das vulnerabilidades que permitisse
			avanços suficientes para uma taxonomia.

			
			Citamos o ano de 2000 como parâmetro, pois nessa época, 1999, surgiria um projeto
			que se tornaria referência para a criação de uma padronização da enumeração
			de vulnerabilidades. Não seria ainda um evento comparável à criação da
			tabela periódica para Química (pois não trouxe a taxonomia) 
			mas certamente lançaria as bases para
			a interoperabilidade exigida para estudos mais aprofundados na área.
			Estamos falando da criação do 
			CVE(Common Vulnerabilities and Exposures)\footnote{http://cve.mitre.org}\footnote{Na época
			de sua criação era originalmente conhecido por Common Vulnerabilities Enumeration - vide
			\cite{Meunier2006} pg. 9.}
			pelo MITRE. A seção \ref{sec:cve} traz mais detalhes.


			Podemos dizer, portanto, que atualmente, embora não tenhamos uma taxonomia
			amplamente aceita pela comunidade, já foi atingido o estágio de enumeração.
			Projetos como o CVE podem ser considerados como marcos dessa etapa.
			A seguir, iremos abordar em mais detalhes o surgimento e o funcionamento dele.
			Isso nos possibilitará compreender melhor a complexidade da classificação
			das vulnerabilidades bem como irá facilitar o entendimento dos capítulos
			seguintes que abordam \textsl{exploits}.
			

	\section{CVE: surgimento e funcionamento}
	\label{sec:cve}
		
	\section{Propostas taxonômicas}

	\section{Métricas para vulnerabilidades}

