
\chapter{Introdução}
\label{chap:introducao}

	\textsl{A Internet...foi projetada no espírito da confiança. Nem os protocolos
		de rede de comunicações nem o software que comanda os sistemas computacionais(nodos)
		conectados a rede foram arquitetados para operação num ambiente 
		no qual estão sob ataque.}\cite{PITAC2005}. 
	Esse trecho\footnote{Traduzido livremente pelo autor do presente trabalho.\label{fn:trad}}
	foi extraído de um \textsl{report} criado pelo comitê consultivo sobre tecnologia da
	informação do presidente dos Estados Unidos. Sua intenção é chamar atenção para
	um problema que ameça todos os países do mundo e que, sem o devido tratamento,
	pode trazer graves consequências às nações.


	Esse mesmo documento vai além e adverte para as consequências da insegurança:
	\textsl{Apesar dos recentes esforços para adicionar componentes de segurança aos
		sistemas computacionais, às redes e ao software, os atos hostis
		se propagam largamente causando danos em escala nacional e internacional.}\footref{fn:trad}
	Conforme os consultores, o elo mais fraco da cadeia é o software.
	\textsl{Os métodos de desenvolvimento de software...falham em prover
		a alta qualidade, a confiabilidade e a segurança das quais os ambientes
		de tecnologia da informação necessitam.}\footref{fn:trad}


	Logo, de acordo com os estudos estratégicos de defesa da nação mais poderosa do mundo,
	um dos seus pontos mais vulneráveis é o software que compõe seus sistemas. E mais,
	o processo de desenvolvimento é considerado como fator predominante para a existência
	dessa ameaça. E, é claro, diante da crescente ubiquidade dos sistemas de computação e, 
	por consequência, do software, aumenta ainda mais a responsabilidade dos projetistas e 
	dos desenvolvedores em garantir que seu trabalho não será utilizado em prejuízo dos usuários.


	Por isso é fundamental que todos aqueles envolvidos no processo de produção
	do software tenham o devido conhecimento das implicações relativas à segurança.
	Conhecer as vulnerabilidades e saber como detectá-las acabam sendo habilidades
	necessárias para garantir projetos seguros.

	
	Outra forma de contribuir para atenuação dessa ameaça, é conhecer as táticas aplicadas
	pelos atacantes. Pois: \textsl{quanto mais se sabe sobre o que o seu inimigo é capaz de fazer,
	maior é a sua condição de discernir sobre quais mecanismos são necessários para sua própria
	defesa}\footref{fn:trad} - \cite{Harris2008}. O conhecimento das técnicas de ataque 
	constitui, portanto, mais uma das dimensões que compõem uma estratégia de defesa.	

	
	O foco do presente trabalho gira em torno de dois conceitos abordados anteriormente:
	vulnerabilidade e \textsl{exploits}. Ambos são apresentados e sob a ótica
	do desenvolvedor de sistemas e do atacante num esforço para contextualizar o leitor
	sobre os caminhos para tornar o software mais seguro.
	
	
	\section{Organização do trabalho}
		Para alcançar o objetivo proposto, esse trabalho está organizado da seguinte forma.
		Inicia com conceitos básicos que servem de suporte aos capítulos seguintes. São
		definições de termos usados e informações básicas sobre o funcionamento do software - como
		organização da memória dos processos, virtualização da memória, entre outros.

		No terceiro capítulo, é abordado o tópico de classificação de vulnerabilidades.
		Nele, o leitor terá contato com propostas e projetos que auxiliam o estudo dessa matéria.


		O quarto capítulo traz uma visão geral sobre técnicas de exploração de vulnerabilidades: os
		\textsl{exploits}.


		Para ilustrar com maior precisão os conceitos de vulnerabilidade e de exploit na prática,
		é apresentado em mais detalhes o \textsl{NULL pointer exploit}. Tópico que assume
		bastante relevância uma vez que uma série de falhas desse gênero foram recentemente
		descobertas - muitas delas no Kernel do Linux, onde permaneceram por mais de 8 anos.


		O último capítulo aborda uma técnica para detecção de vulnerabilidades: \textsl{Fuzzing}.
		Foi escolhida entre as demais para constar nesse trabalho pois possui excelente relação
		custo benefício. É destacada com o intuito de mostrar aos desenvolvedores estejam
		atentos aos métodos empregados pelos atacantes e possam aplicá-las deles.
		
		

		

		

