
\chapter{Conclusão}
\label{chap:conclusao}
	Nesse trabalho foram abordados aspectos essenciais relacionados à segurança do software.
	Ao tratar de vulnerabilidades e técnicas de \textsl{exploits}, ele objetivou
	trazer ao leitor um contexto fundamental para um entendimento da área. Dada a relevância
	que o software atingiu nos dias de hoje, não é mais admissível que qualquer desenvolvimento
	sério desconsidere princípios de segurança. Sendo eles: as possíveis vulnerabilidades,
	as formas de ataque e, naturalmente, as formas de prevenção.


	No que se refere às vulnerabilidades, esse trabalho, ao tratar de sua classificação, pode identificar
	que esse tópico ainda não é pacífico no meio acadêmico ou industrial. Mesmo que tenham
	sidos feitos avanços, a comunidade carece de um padrão aceito uniformemente. Ficou nítido que a complexidade
	dessa tarefa é enorme. O próprio caráter multifacetado das vulnerabilidades explica um pouco essa barreira;
	elas podem ser analisados por diversos ângulos e estamos longe de encontrar uma visão unificadora
	que traga sentido a todas suas faces. Apenas assim seria alcançada uma taxonomia em sentido estrito.


	No outro eixo fundamental, dos \textsl{exploits}, pode ser visto que, com a evolução natural
	das técnicas de ataque e de defesa, a vida dos especialistas na área torna-se cada vez mais árdua.
	As formas mais simples de explorar vulnerabilidades já não são mais efetivas; seja porque as falhas
	que as tornam possíveis ficaram menos frequentes no desenvolvimento, seja porque proteções mais
	bem concebidas foram sendo habilitadas por padrão nos sistemas. Isso vai obrigando os atacantes
	a encontrarem novos métodos cada vez mais sofisticados que, naturalmente, vão exigindo conhecimento
	ainda mais específico. Em alguns momentos, porém, ainda será possível surgir alguma espécie
	de reviravolta - como na descoberta dos diversos erros de \textsl{NULL pointer} no kernel do Linux.
	Episódio que demonstrou a existência de uma série de falhas por vários anos em um dos sistemas
	mais utilizados - surpreendentemente, algumas delas de fácil exploração.


	Para a prevenção de problemas de segurança no software, uma das principais propostas apresentadas foi
	o uso do testes fuzzing. Sendo uma técnica extremamente eficiente e que já vem sendo usada
	por atacantes para a descoberta de problemas nos sistemas, esse trabalho buscou demonstrar seu valor
	e indicá-la como arma a ser utilizada pelos próprios desenvolvedores. Por que já não conceber, desde
	o princípio, um projeto considerando essa alternativa de teste se os atacantes certamente irão utilizá-la?
	Conforme foi visto, gigantes da área, como a Microsoft, já perceberam seu enorme valor e investem
	fortemente nela. É necessário, portanto, que, ao menos, consideremos essa possibilidade. Isso
	porque não é aceitável correr o risco de deixar apenas para os atacantes utilizarem e aperfeiçoarem
	uma técnica que possa desequilibrar em favor deles.

