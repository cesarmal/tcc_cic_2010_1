
\chapter{NULL pointer Exploit}
\label{chap:null_pointer_exploit}

	Dentre as várias técnicas de exploits existentes, uma que certamente merece
	destaque, é o NULL pointer exploit.
	Surgiu recentemente e é fruto da crescente dificuldade em aplicar técnicas
	que exploram vulnerabilidades de corrupção de memória.
	Em princípio, deferenciar um ponteiro nulo não é considerada
	uma vulnerabilidade explorável, mas, como veremos, isso nem sempre é verdade.

	
	O ano de 2009 chegou a ser considerado o ano do "kernel NULL pointer deference"
	em virtude da grande quantidade de falhas desse gênero encontradas no kernel do Linux.
	(citar http://cwe.mitre.org/top25/ para o ano de 2009)
	Vemos então que essa falha chegou a atingir notoriedade pelo grande impacto que causou.
	Nossa intenção é apresentar esse tipo de exploit, mostrar como pode ocorrer, ilustrando
	com exemplos e apontando formas de prevenção.
	
	
	
	\section{O que é um NULL pointer}
		O primeiro ponto a ser abordado é o NULL pointer.
		Na linguagem de programação C, podemos considerar um ponteiro como um valor inteiro
		que referencia uma posição de memória. Ou seja, trata-se de um valor que aponta
		para um determinado ponto no espaço de endereçamento. Quando um ponteiro é deferenciado,
		passamos a acessar o valor presente na posição de memória para o qual ele aponta.
		Ilustrando, segue pequeno trecho de código C.
		\begin{lstlisting}[label=pointer_example,caption=Ponteiro em C]
int val = 10;
int *pointer = &value;
/* pointer has the address of val */
/* *pointer returns the value stored in val */
		\end{lstlisting}
		

		No Linux, o arquivo stddef.h contém a definição de NULL, que por
		convenção, denomina um ponteiro com valor zero.
		Um ponteiro nulo, então, aponta para a posição zero de memória. 
		Como, em regra geral, os sistemas utilizam o esquema de memóra virtual,
		na prática, esse endereço zero deve ser considerado tão somente no espaço
		de endereçamento do processo em questão.
		Como normalmente ele não constitui um mapeamento válido, pois os processos não
		iniciam com aquela porção mapeada, os acessos a essa região implicam violação
		às regras do esquema de memória virtual. Erros desse gênero resultam em
		segmentation fault.   

		Logo, quando deferenciamos um ponteiro nulo, temos um acesso não previsto
		que configura uma violação 
		Por essa razão, que o acesso a um ponteiro com valor
		zero acaba sendo um erro na aplicação que resulta em sua finalização.
		
		Embora o foco do trabalho recaia sobre a arquitetura x86, é válido identificar a repercussão
		de uma acesso a posição zero de memória em outros casos.
		Existem arquiteturas nas quais o endereço zero já é mapeado inicialmente. 
		Podemos apontar o caso da ARM e da XScale; ambas para sistemas embarcados. 
		Nelas, o vetor de exceções se encontra nessa posição. Ele contém
		endereços para o tratamento de, por exemplo, interrupções de hardware e software.
		Naturalmente, um acesso a um ponteiro nulo nesses casos constitui enorme
		problema em termos de segurança.

		

	\section{Como funciona a técnica}

	\section{Exemplos}

	\subsection{Flash Player exploit}
	
	\subsection{Falhas no kernel do Linux}
		Existem diversas falhas no kernel do Linux relacionadas a NULL pointer.
		Desde problemas na inicialização de estruturas de dados até a erros na compilação.		
	
	\subsection{Pidgin}

	\section{Proteções}
